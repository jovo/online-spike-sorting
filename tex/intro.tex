
The recently announced BRAIN initiative calls for the development of technologies that will advance our understanding of neuroscience \cite{??}.  Crucial to the success of this endeavor will be the advancement of our ability to understand the \emph{dynamics} of the brain, via the measurement of large populations of neural activity at the single neuron level.  Such reverse engineering efforts will benefit from real-time decoding of neural activity, to facilitate adapting the stimuli to probe the functional connectivity more effectively.  Regardless of the experimental apparati used to measure such data (e.g., electrodes or calcium imaging), real-time decoding of individual neuron responses requires real-time spike sorting from large populations of neurons.

Automatic spike sorting methods are continually evolving to deal with more sophisticated experiments.  Most recently, several methods have been proposed to (i) learn the number of separable single units on each electrode or ``multi-trode'' \cite{??}, or (ii) operate online to resolve overlapping spikes from multiple neurons \cite{??}.   To our knowledge, no method to date is able to simultaneously address both of these challenges.  

We develop a fully Bayesian continuous-time generative model of population activity.  More specifically, we posit the existence of a latent marked Poisson process for each neuron.  Previous efforts to address overlapping spiking often assume a fixed kernel for each waveform, but joint intracellular and extracellular recording indicate that this assumption is clearly violated (see Figure \ref{fig:draft}). Thus, we assume that the statistics of the ``mark'' of each Poisson process are time-varying.  Moreover, rather than assuming a priori a fixed number of separable neurons per channel, we place a nonparametric Bayesian prior on the number of neurons.  Collectively we therefore have a completely random measure jointly characterizing spike times and waveforms from a potentially infinite number of neurons \cite{??}.  

Given the above infinite dimensional continuous time stochastic process, we show that it is the limiting process of a discrete time model.  We then develop an online variational Bayesian inference algorithm for this model \cite{??}.  Via numerical simulations, we demonstrate that our inference procedure improves over the previous state-of-the-art, even though we allow the other methods to use the entire dataset for training, whereas we learn online.  Moreover, we demonstrate that we can effective track the time-varying changes in waveform, and detect overlapping spikes.  Indeed, it seems that the false positives detections from our approach have indistinguishable first order statistics from the true positives, suggesting that that second-order methods may be required to reduce false positive rate (i.e., template methods may be inadequate).  Our work therefore suggests that further improvements in real-time decoding of activity may be most effective if directed at simultaneous real-time spike sorting and decoding.  To facilitate such developments and support reproducible research, all code and data associated with this work is provided in the Supplementary materials.