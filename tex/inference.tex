\newcommand{\tx}{\tilde{x}}

We now address the problem of posterior inference over the latent variables gives the matrix $\bX$ of multielectrode recordings. Unsurprisingly, exact 
inference is intractable, and we have to resort to approximating the posterior distribution.
There exists a vast literature on approximate inference for nonparametric models, especially so for models based on the Dirichlet process.
Traditional approaches are sampling-based, typically involving Markov chain Monte Carlo techniques (see eg.\ \citep{Nea2000, IshJam2001}), 
while recently there has also been work on constructing deterministic approximations to the intractable posterior (\citep{BleJor2006, MinGha2003}).
Our problem is complicated by two factors. The first is the convolutional nature of our observation process, where we observe a functional of the 
sequence of observations drawn from the DP mixture model. This is in contrast to the usual situation where one directly observes the DPMM outputs themselves.
The second complication is a computational requirement: typical inference schemes are batch methods that are slow and computationally expensive. 
Our goal, on the other hand, is ultimately to perform inference in real time, so that these batch approaches unsuitable for our purposes.

With the latter objective in mind, we develop an online algorithm for posterior inference. Our algorithm is based on \cite{WangDun2009}, though this was 
developed for the usual situation where one has i.i.d.\ observations drawn from a DPMM. We generalize this algorithm to our observation process, and to 
account for the time-evolution of the cluster parameters and the Markov nature of the observation noise.

At a high level, at time $t$, our inference algorithm maintains a set of the times of spikes underlying the observations until time $t$. It also maintains
the identities of the neurons those spikes were assigned to, as well as the weight vectors determining the shapes of the associated spike waveforms.
Besides these point estimates, we also maintains a set of posterior distributions $q_{it}(\theta^*_i)$ where $i$ spans over the
set of neurons associated with the spikes seen so far. For each $i$, $q_{it}(\theta^*_i)$ approximates the distribution over the parameters 
$\theta_i^* \equiv (\mu_i^*, \Sigma_i^*)$ of neuron $i$ given the observations until time $t$. 
Having identified the location and shape of spikes from earlier times, we can calculate their contribution to the observation at time $t+1$.
We subtract this from the actual observation at time $t+1$, and treat the residual $\tx_{t+1}$ as an observation from a DP mixture model.
Given this, our algorithm then makes a hard decision about whether a spike is present, whether that spike belongs to one of the earlier neurons
or a new neuron, and what the shape of the associated spike waveform is. We use this to recursively update the distribution over neuron parameters.
We describe this recursive algorithm below.


\begin{align}
  \tx_t = x_t - \sum_{i=1}^L \bA\by_{t-i} \label{eq:resid}
\end{align}
Given the spike statistics from earlier times, as well as the distribution over parameters, our algorithm decides whether there is a spike underlying
the observations at time $t$, which cluster it is assigned to, and what
the shape of that spike is. We simultaneously update the distribution over parameters of clusters. 

Recall that each spike waveform spans $L$ time bins. 
Lett $z_t$ indicate whether or not a spike is present at time $t$, and with $\by_t$ giving its shape. Equation \eqref{eq:resid} gives the residual 
at time $t$, we must first decide whether or not there is a spike underlying this residual.
By Bayes' rule,
\begin{align}
  P(z_t = 1 | \tx_t)  &\propto P(z_t = 1, \tx_t) = \sum_{\gamma_t = 1}^{C_t} P(\tx_t, {\gamma_t} | z_t = 1) P(z_t = 1) \\
  P(\tx_t | {\gamma_t} , z_t = 1) &= \int_{\Theta} P(\tx_t | \theta_t) q_{\gamma_t} (\theta_t | \gamma_t) \dd \theta_t
\end{align}
Having calculated $P(z_t = 1 | \tx_t)$, we decide that there is a spike present if this exceeds a threshold of $0.5$, otherwise, we set $z_t = 0$.
In the event that we decide that a spike is present (after marginalizing over all possible cluster assignments), we simplify matters by assigning 
the spike to the cluster with highest posterior probability. Similarly, we set the weight vector $\by_t$ to its MAP value.
Given these two, we update the posterior distribution over parameters of the cluster recursively calculating $q_{t+1}$.
