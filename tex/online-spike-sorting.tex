\documentclass{article} % For LaTeX2e
\usepackage{nips13submit_e,times}
\usepackage{hyperref}
\usepackage{url}
\usepackage{natbib}
\usepackage{comment}
\usepackage{amssymb, amsmath}
%\documentstyle[nips13submit_09,times,art10]{article} % For LaTeX 2.09


\title{Formatting Instructions for NIPS 2013}


\author{
David S.~Hippocampus\thanks{ Use footnote for providing further information
about author (webpage, alternative address)---\emph{not} for acknowledging
funding agencies.} \\
Department of Computer Science\\
Cranberry-Lemon University\\
Pittsburgh, PA 15213 \\
\texttt{hippo@cs.cranberry-lemon.edu} \\
\And
Coauthor \\
Affiliation \\
Address \\
\texttt{email} \\
\AND
Coauthor \\
Affiliation \\
Address \\
\texttt{email} \\
\And
Coauthor \\
Affiliation \\
Address \\
\texttt{email} \\
\And
Coauthor \\
Affiliation \\
Address \\
\texttt{email} \\
(if needed)\\
}

% The \author macro works with any number of authors. There are two commands
% used to separate the names and addresses of multiple authors: \And and \AND.
%
% Using \And between authors leaves it to \LaTeX{} to determine where to break
% the lines. Using \AND forces a linebreak at that point. So, if \LaTeX{}
% puts 3 of 4 authors names on the first line, and the last on the second
% line, try using \AND instead of \And before the third author name.

\newcommand{\fix}{\marginpar{FIX}}
\newcommand{\new}{\marginpar{NEW}}
\newcommand{\bX}{\mathbf{X}}
\newcommand{\bx}{\mathbf{x}}
\newcommand{\dd}{\mathrm{d}}
\newcommand{\Levy}{L\'{e}vy }
%\nipsfinalcopy % Uncomment for camera-ready version

\begin{document}


\maketitle

\begin{abstract}
\end{abstract}

\section{The model/introduction}
Our data is a time-series of multielectrode recordings $\bX \equiv (\bx_1, \cdots, \bx_T)$, and consists of $T$ recordings from $C$ channels. 
The set of recording times lie on regular grid with interval length $\Delta$, while $\bx_t \in \mathbb{R}^C$ for all $t$. 
This time-series of electrical activity is driven by an unknown number of neurons {\color{red} and we want to... outline scientific goals}. 
We let the number of neurons be unbounded, though only a few of the infinite
neurons dominate. These neurons contribute the majority of the activity in any finite interval of time; however, as time passes, the total number of 
observed neurons increases {\color{red}(Justify?)}. 
%Each neuron, has its own `shape' A natural model in such a situation is to
The neurons themselves emit continuous-time voltage traces, with the outputs of all neurons superimposed and discretely sampled to produce the 
recordings $\bX$.  At a high level, we model the output of each neuron as a
series of idealized spikes smoothed with appropriate kernels (the latter determines the shape of each action potential). 
%Each neuron has its own distribution over waveform shapes. 
We describe this in detail, starting first with the model for a single channel recording $X \equiv (x_1, \cdots, x_T)$.

\subsection{Modelling a single neuron output}
We model the spiking activity of each neuron as stationary and memoryless, with its set of spike times distributed as a homogeneous Poisson process. 
{\color{red} Comment on refractoriness or leave for
discussion/future work. Similarly on the generalization to inhomogeneity?} The neurons themselves are heterogeneous, with $r_i$ the (unknown) firing 
rate for neuron $i$. Call the ordered set of spike times of the $i$th neuron $E_i$; then the time between successive elements of $E_i$ is exponentially 
distributed with mean $1/r_i$. 
We write this as
\begin{align}
  E_i &\sim \text{PoissProc}(r_i)
\end{align}
The actual electrical output of a neuron is not a binary event; instead each spiking event is a smooth voltage perturbation about a
resting state. This perturbation forms the shape of the spike (without any loss of generality, we set the resting state to zero). 
{(\color{red} figure? better biological description? comment on how we preprocess the data to get zero mean?)}. 
While the spike shape varies across neurons as well as across different spikes of the same neuron, each 
neuron has its own characteristic distribution over shapes. {\color{red} Figure? } 
We let $\theta \in \Theta$ parametrize this distribution, and whenever neuron $i$ emits a 
spike, we draw a voltage trace independently from the corresponding distribution. %$p_{\theta_i}$, and 
This is then offset to the time of the spike, and the complete output of the neuron is the superposition of all these spike waveforms. 
{\color{red} ( Figure?)} % Comment on how this dictionary is obtained now, or in section on inference?)}. 
More concretely, we model each spike shape as a linear combination of a dictionary of $K$ basis functions $A \equiv (A_1(t), \cdots, A_K(t))$, shared across 
all neurons.
For the $i$th neuron, the $j$th spike $e_{ij} \in E_i$, is associated with a random $K$-dimensional weight vector $\tilde{y}_{ij}$, and the 
shape of this spike is given by the weighted sum $\sum_{k=1}^K \tilde{y}_{ijk} A_k(t)$. We let $\tilde{y}_{ij}$ be Gaussian distributed, with 
$\theta_i \equiv (\mu_i, \Sigma_i)$ determining its 
mean and variance. Then, at any time $t$, the output of neuron $i$ is
\begin{align}
  x_{i}(t) &= \sum_{j=1}^{|E_i|} \sum_{k=1}^K \tilde{y}_{ijk} A_k(t - e_{ij})
\end{align}

{The total signal recorded $x(t)$ at any electrode is the superposition of the outputs of all neurons. Define $E = \cup_{i=1}^{\infty} E_i$ as
the (ordered) superposition of the spike times of all neurons. 
Furthermore, let $n(j)$ be the neuron to which the $j$th element of $E$ belongs, and let $p(j)$ index the position of the $j$th spike of $E$ in the spike train
$E_{n(i)}$ of neuron $n(i)$ (so that $e_j = e_{n(j)p(j)}$). Then, we have that}
\begin{align}
  x(t) &= \sum_{i=1}^{\infty} x_{i}(t) =   \sum_{j=1}^{|E|} \sum_{k=1}^K y_{jk} A_k(t - e_{j}) \label{eq:spk_sup}
\intertext{where}
  y_{j} &\equiv \tilde{y}_{n(j)p(j)} \sim N(\mu_{n(j)}, \sigma_{n(j)}) \label{eq:spk_shape}
\end{align}

From the superposition property of the Poisson process \citep{kingman93}, the overall spiking activity $E$ is a 
Poisson process with rate $R = \sum_{i=1}^{\infty} r_i$. The signal $x(t)$ is a functional of a marked Poisson process, where the $j$th event
is labelled by the neuron to which it is assigned ($n(j)$), and the shape of its spike waveform ($y_j$). From the properties of the Poisson
process, we have that the marks $n(j)$ are i.i.d. distributed with $P(n(j) = i) = \frac{r_i}{R}$. Given $n(j)$, $y_j$ is distributed as in
equation \ref{eq:spk_shape}.

\subsection{Completely random measures (CRMs)}
In this work, we take a nonparametric approach, letting the number of neurons be unbounded (so that $n(i) \in \{1, 2, \cdots \}$).
Since only a finite number of spikes are observed in any finite interval, the total rate $R$ must 
also be finite; moreover, as we described earlier, we want this to be dominated by a few $r_i$. 
A natural framework that captures these  modelling requirements is that of completely random measures \citep{Kingman:PJM67}.
Completely random measures are stochastic processes that form flexible and convenient priors over
infinite dimensional objects like probability distributions, hazard functions etc. 
These have been well studied in the Bayesian nonparametrics and machine learning community, and there exists a wealth of literature on
their theoretical properties, as well as on computational approaches to posterior inference.

Recall that each neuron is characterized by a pair $(r_i, \theta_i)$; the former characterizes the distribution over spike times, and the latter over spike
shapes. We map the infinite collection of pairs $\{(r_i, \theta_i)\}$ to an atomic measure on $\Theta$:
\begin{align}
  R(\dd \theta) = \sum_{i=1}^{\infty} r_i \delta_{\theta_i}
\end{align}
For any subset $\varTheta$ of $\Theta$, the measure $R(\varTheta)$ equals \( \sum_{\{ i: \theta_i \in \varTheta \} } r_i\). We allow $R(\cdot)$ to be random,
modelling it as a realization of a completely random measure. Such a random measure has the property that for any two disjoint subsets $\varTheta_1$ 
and $\varTheta_2 \in \Theta$, the measures $R(\varTheta_1)$ and $R(\varTheta_2)$ are independent. 
This distribution over measures is induced by a distribution
over the infinite sequence of weights (the $r_i$'s), and a distribution over the sequence of their locations (the $\theta_i$'s). 
For a CRM, the weights $r_i$ form the jumps of a \Levy process \citep{Sato90}, and their distribution is characterized by a 
\Levy intensity $\rho(r)$. The locations $\theta_i$ are drawn i.i.d.\  from a base probability measure $H(\theta)$; we let this be the conjugate
normal-Wishart distribution. As it typical, we assume these to be independent (though this is not necessary). {\color{red} if there's space, I
can elaborate on the construction of the CRM from its Levy measure, though this is not necessary}

The CRM we choose is the Gamma process ($\Gamma$P); this has \Levy intensity $\rho(r) = r^{-1}\exp(-r\alpha)$. The Gamma process has the convenient property that the 
total mass $R \equiv R(\Theta) = \sum_{i=1}^{\infty} r_i$ is Gamma distributed (and thus conjugate to the Poisson process prior on $E$). 
%The Gamma distribution has shape parameter $1$ and scale parameter $\alpha$.  Since this is finite almost surely, so too is $E$. 
The Gamma process is also closely connected with the Dirichlet process \citep{Ferguson73}, which will prove useful
later on.
Other choices of the \Levy intensity can capture greater uncertainty in the number of neurons active in any finite interval, power-law behaviour etc.
In any case, our overall model is then:
\begin{align}
  R(\dd \theta) & \sim \Gamma \text{P}(\alpha, H(\theta)) \\ %\mathcal{NW}(\mu, \Sigma)) \\
  E_i\ \  &\sim \text{PoissProc}(r_i) \quad i \text{ in } 1,2,\cdots \\
  \tilde{y}_{ij} & \sim N(\mu_i, \Sigma_i) \quad i,j \text{ in } 1,2,\cdots \\
  x_i(t) &= \sum_{j = 1}^{|E_i|} \tilde{y}_{ij} A_j(t - e_{ij}) \\
  X   &= \sum_{i=1}^{\infty} x_i
\end{align}

%Each spike of each neuron is associated with a time $e$ and a weight vector $y$, and one can view the model above as a doubly stochastic Poisson
%process on the product space. 

It will be more convenient from the point of inference to work with the marked Poisson process representation of equations \ref{eq:spk_sup} and \ref{eq:spk_shape}. 
The superposition process $E$ is a rate $R$ Poisson process,
and under a Gamma process prior, $R$ has a conjugate Gamma distribution with shape and scale parameters $1$ and $\alpha$ respectively.
As we saw, the labels $n(\cdot)$ assigning events to neurons are drawn i.i.d. from a normalized Gamma process $G(\dd \theta)$:
\begin{align}
 G(\dd \theta) = \frac{r_j}{R}
\end{align}
$G(\dd \theta)$ is a random probability measure that belongs to a class called a normalized random measures \citep{JamesLP09}; for the Gamma process, 
this is a draw from the Dirichlet process. For the $j$ spike, given its neuron assignment $n(i)$, its shape vector is drawn from a normal distribution
with parameters $(\mu_{n(j)}, \Sigma_{n(j)})$. Thus the weight vectors are distributed according to a Dirichlet process mixture model, with
the neurons forming clusters. This insight allows us to marginalize out the infinite-dimensional rate vector $R$, and assign spikes to neurons via
the Chinese restaurant process (CRP). Under the CRP, the $j$th spike is assigned to one of the earlier neurons with probability proportional to the number
of earlier spikes assigned to that neuron. It is assigned to a new neuron with probability $\alpha$.
Unlike most applications with observe the outputs of a CRP, we observe a functional of it.
%is assigned (or equivalently, the parameter $\theta$ associated with that neuron). These marks are drawn from a probability measure 
%$G(\dd \theta) = \frac{1}{R} R(\dd \theta)$. From the properties of the Gamma process, the probability measure $G$ a Dirichlet process, 
Furthermore, (again, for the Gamma process), the random probability measure $G$ is independent of the total mass $R(\Theta)$. We thus have
 the following model equivalent to the one above:
\begin{align}
  R & \sim \text{Gamma}(1,\alpha) \\
  G(\dd \theta) & \sim \text{DP}(\alpha) \\
  E &\sim \text{PoissProc}(R) \\
  n(j) &\sim G, \quad j = 1,\cdots, |E| \\
  y_e &\sim N(\mu_{n(e)}, \Sigma_{n(e)}), \quad  j = 1,\cdots, |E| \\
  x(t) &=   \sum_{j=1}^{|E|} \sum_{k=1}^K y_{jk} A_k(t - e_{j})
\end{align}

%For neuron $i$, the sequence of spike times is distributed as a Poisson process with random rate $r_i$.
%Each event $e_{ij}$ is associated with a mark or label $y_{ij}$ drawn from a normal distribution (again, with random parameters).
%More broadly, we can view the superposed process $E$ as a rate $R$ Poisson process, with each event having a pair of marks, the neuron identity $i$,
%and weight $y$. From the properties of the Gamma process, the pair form a draw from a Dirichlet process.
%Our data is in a form that makes discrete-time modelling more natural, and an approach now is
%one based on the Beta process-binomial process.

\subsection{A discrete-time approximation}
In the previous paragraphs, we described a continuous-time voltage output by a neuron. Our data on the other hand consists of recordings
at a discrete set of times. While it is possible to make inferences about the continuous-time process that underlies these discrete recordings,
in this work we restrict ourselves to discrete-time inferences. Towards this, we start by providing a discrete-time approximation to the model above. 
This follows easily from the marked Poisson process characterization of the model.
Recall first the Bernoulli approximation to the Poisson process: a sample from a Poisson process with rate $R$ can be approximated by discretizing
time at a granularity $\Delta$, and assigning each interval an event independently with probability $R\Delta$. This approximation becomes exact
as $\Delta$ tends to $0$.

This suggests the following approximation at a time resolution $\Delta$. Draw the random Poisson process rate $R$ drawn from a Gamma$(1,\alpha)$ 
distribution. Simultaneously, draw a random probability measure
 $G$ from a Dirichlet process. Assign an event to an interval independently with probability $R\Delta$, and to each event, assign a random mark drawn 
from the DP. Given the marks, we can evaluate the recordings at each time.

\subsection{Noise and nonstationarity}
The signal recorded by an electrode is the neuron output corrupted by noise, we model this noise as independent of the signal, additive and Gaussian.
However, rather than modelling the noise as independent across time bins, we model it as a first-order autoregressive process. This can capture
effects like the movement of electrodes during the experiment. Furthermore, rather than keeping the cluster parameters fixed, we model these as
AR processes as well, capturing the evolution of the neuron shape with time.

\subsection{Modelling multielectrode recordings}

\section{Inference}
We perform inference in an online manner \cite{WangDun2009}. As observations arrive, our inference algorithm decides whether or not a 
new spike is present, which neuron (cluster) to assign that spike to, as well as the shape of the spike waveform. On the other hand, our algorithm 
maintains a posterior distribution over the cluster parameters that characterize the distribution over shapes. Having identified the location and shape of 
spikes from earlier times, we subtract these from the observations treat the residual as an observation from a DP mixture model.
The cluster assignment of earlier spikes determines the seating arrangement of customers in the Chinese restaurant associated with the DP. Given the
corresponding distribution over parameters, $p(\theta)$, we decide whether there is an underlying spike, which cluster it is assigned to, and what
the shape of that spike is. We simultaneously update the distribution over parameters of clusters. Assume each spike waveform spans $W$ time intervals. 
Define the residual at time $t$ as $X_t - \sum_{i=1}^W A$. At time $t$, let $y_t$ represent the shape of the action potential.
Letting $\tilde{x}_t$ be the observation at time $t$, we have
\begin{align}
  z_t  & \sim Bern(p) \\
  \intertext{\hspace{2in} if $z_t == 1$}
  \gamma_t| \gamma_{1:t-1} &\sim CRP \\
  \theta_t| \gamma_t = i & \sim \mathcal{N}(\theta_{t-1},\Sigma)
\end{align}

\bibliography{refs}
\bibliographystyle{unsrt}
\begin{comment}
\section{Submission of papers to NIPS 2013}
NIPS requires electronic submissions.  The electronic submission site is  
\begin{center}
   \url{http://papers.nips.cc}
\end{center}

Please read carefully the
instructions below, and follow them faithfully.
\subsection{Style}

Papers to be submitted to NIPS 2013 must be prepared according to the
instructions presented here. Papers may be only up to eight pages long,
including figures. Since 2009 an additional ninth page \textit{containing only
cited references} is allowed. Papers that exceed nine pages will not be
reviewed, or in any other way considered for presentation at the conference.
%This is a strict upper bound. 

Please note that this year we have introduced automatic line number generation
into the style file (for \LaTeXe and Word versions). This is to help reviewers
refer to specific lines of the paper when they make their comments. Please do
NOT refer to these line numbers in your paper as they will be removed from the
style file for the final version of accepted papers.

The margins in 2013 are the same as since 2007, which allow for $\approx 15\%$
more words in the paper compared to earlier years. We are also again using 
double-blind reviewing. Both of these require the use of new style files.

Authors are required to use the NIPS \LaTeX{} style files obtainable at the
NIPS website as indicated below. Please make sure you use the current files and
not previous versions. Tweaking the style files may be grounds for rejection.

%% \subsection{Double-blind reviewing}

%% This year we are doing double-blind reviewing: the reviewers will not know 
%% who the authors of the paper are. For submission, the NIPS style file will 
%% automatically anonymize the author list at the beginning of the paper.

%% Please write your paper in such a way to preserve anonymity. Refer to
%% previous work by the author(s) in the third person, rather than first
%% person. Do not provide Web links to supporting material at an identifiable
%% web site.

%%\subsection{Electronic submission}
%%
%% \textbf{THE SUBMISSION DEADLINE IS MAY 31st, 2013. SUBMISSIONS MUST BE LOGGED BY
%% 23:00, MAY 31st, 2013, UNIVERSAL TIME}

%% You must enter your submission in the electronic submission form available at
%% the NIPS website listed above. You will be asked to enter paper title, name of
%% all authors, keyword(s), and data about the contact
%% author (name, full address, telephone, fax, and email). You will need to
%% upload an electronic (postscript or pdf) version of your paper.

%% You can upload more than one version of your paper, until the
%% submission deadline. We strongly recommended uploading your paper in
%% advance of the deadline, so you can avoid last-minute server congestion.
%%
%% Note that your submission is only valid if you get an e-mail
%% confirmation from the server. If you do not get such an e-mail, please
%% try uploading again. 


\subsection{Retrieval of style files}

The style files for NIPS and other conference information are available on the World Wide Web at
\begin{center}
   \url{http://www.nips.cc/}
\end{center}
The file \verb+nips2013.pdf+ contains these 
instructions and illustrates the
various formatting requirements your NIPS paper must satisfy. \LaTeX{}
users can choose between two style files:
\verb+nips11submit_09.sty+ (to be used with \LaTeX{} version 2.09) and
\verb+nips11submit_e.sty+ (to be used with \LaTeX{}2e). The file
\verb+nips2013.tex+ may be used as a ``shell'' for writing your paper. All you
have to do is replace the author, title, abstract, and text of the paper with
your own. The file
\verb+nips2013.rtf+ is provided as a shell for MS Word users.

The formatting instructions contained in these style files are summarized in
sections \ref{gen_inst}, \ref{headings}, and \ref{others} below.

%% \subsection{Keywords for paper submission}
%% Your NIPS paper can be submitted with any of the following keywords (more than one keyword is possible for each paper):

%% \begin{verbatim}
%% Bioinformatics
%% Biological Vision
%% Brain Imaging and Brain Computer Interfacing
%% Clustering
%% Cognitive Science
%% Control and Reinforcement Learning
%% Dimensionality Reduction and Manifolds
%% Feature Selection
%% Gaussian Processes
%% Graphical Models
%% Hardware Technologies
%% Kernels
%% Learning Theory
%% Machine Vision
%% Margins and Boosting
%% Neural Networks
%% Neuroscience
%% Other Algorithms and Architectures
%% Other Applications
%% Semi-supervised Learning
%% Speech and Signal Processing
%% Text and Language Applications

%% \end{verbatim}

\section{General formatting instructions}
\label{gen_inst}

The text must be confined within a rectangle 5.5~inches (33~picas) wide and
9~inches (54~picas) long. The left margin is 1.5~inch (9~picas).
Use 10~point type with a vertical spacing of 11~points. Times New Roman is the
preferred typeface throughout. Paragraphs are separated by 1/2~line space,
with no indentation.

Paper title is 17~point, initial caps/lower case, bold, centered between
2~horizontal rules. Top rule is 4~points thick and bottom rule is 1~point
thick. Allow 1/4~inch space above and below title to rules. All pages should
start at 1~inch (6~picas) from the top of the page.

%The version of the paper submitted for review should have ``Anonymous Author(s)'' as the author of the paper.

For the final version, authors' names are
set in boldface, and each name is centered above the corresponding
address. The lead author's name is to be listed first (left-most), and
the co-authors' names (if different address) are set to follow. If
there is only one co-author, list both author and co-author side by side.

Please pay special attention to the instructions in section \ref{others}
regarding figures, tables, acknowledgments, and references.

\section{Headings: first level}
\label{headings}

First level headings are lower case (except for first word and proper nouns),
flush left, bold and in point size 12. One line space before the first level
heading and 1/2~line space after the first level heading.

\subsection{Headings: second level}

Second level headings are lower case (except for first word and proper nouns),
flush left, bold and in point size 10. One line space before the second level
heading and 1/2~line space after the second level heading.

\subsubsection{Headings: third level}

Third level headings are lower case (except for first word and proper nouns),
flush left, bold and in point size 10. One line space before the third level
heading and 1/2~line space after the third level heading.

\section{Citations, figures, tables, references}
\label{others}

These instructions apply to everyone, regardless of the formatter being used.

\subsection{Citations within the text}

Citations within the text should be numbered consecutively. The corresponding
number is to appear enclosed in square brackets, such as [1] or [2]-[5]. The
corresponding references are to be listed in the same order at the end of the
paper, in the \textbf{References} section. (Note: the standard
\textsc{Bib\TeX} style \texttt{unsrt} produces this.) As to the format of the
references themselves, any style is acceptable as long as it is used
consistently.

As submission is double blind, refer to your own published work in the 
third person. That is, use ``In the previous work of Jones et al.\ [4]'',
not ``In our previous work [4]''. If you cite your other papers that
are not widely available (e.g.\ a journal paper under review), use
anonymous author names in the citation, e.g.\ an author of the
form ``A.\ Anonymous''. 


\subsection{Footnotes}

Indicate footnotes with a number\footnote{Sample of the first footnote} in the
text. Place the footnotes at the bottom of the page on which they appear.
Precede the footnote with a horizontal rule of 2~inches
(12~picas).\footnote{Sample of the second footnote}

\subsection{Figures}

All artwork must be neat, clean, and legible. Lines should be dark
enough for purposes of reproduction; art work should not be
hand-drawn. The figure number and caption always appear after the
figure. Place one line space before the figure caption, and one line
space after the figure. The figure caption is lower case (except for
first word and proper nouns); figures are numbered consecutively.

Make sure the figure caption does not get separated from the figure.
Leave sufficient space to avoid splitting the figure and figure caption.

You may use color figures. 
However, it is best for the
figure captions and the paper body to make sense if the paper is printed
either in black/white or in color.
\begin{figure}[h]
\begin{center}
%\framebox[4.0in]{$\;$}
\fbox{\rule[-.5cm]{0cm}{4cm} \rule[-.5cm]{4cm}{0cm}}
\end{center}
\caption{Sample figure caption.}
\end{figure}

\subsection{Tables}

All tables must be centered, neat, clean and legible. Do not use hand-drawn
tables. The table number and title always appear before the table. See
Table~\ref{sample-table}.

Place one line space before the table title, one line space after the table
title, and one line space after the table. The table title must be lower case
(except for first word and proper nouns); tables are numbered consecutively.

\begin{table}[t]
\caption{Sample table title}
\label{sample-table}
\begin{center}
\begin{tabular}{ll}
\multicolumn{1}{c}{\bf PART}  &\multicolumn{1}{c}{\bf DESCRIPTION}
\\ \hline \\
Dendrite         &Input terminal \\
Axon             &Output terminal \\
Soma             &Cell body (contains cell nucleus) \\
\end{tabular}
\end{center}
\end{table}

\section{Final instructions}
Do not change any aspects of the formatting parameters in the style files.
In particular, do not modify the width or length of the rectangle the text
should fit into, and do not change font sizes (except perhaps in the
\textbf{References} section; see below). Please note that pages should be
numbered.

\section{Preparing PostScript or PDF files}

Please prepare PostScript or PDF files with paper size ``US Letter'', and
not, for example, ``A4''. The -t
letter option on dvips will produce US Letter files.

Fonts were the main cause of problems in the past years. Your PDF file must
only contain Type 1 or Embedded TrueType fonts. Here are a few instructions
to achieve this.

\begin{itemize}

\item You can check which fonts a PDF files uses.  In Acrobat Reader,
select the menu Files$>$Document Properties$>$Fonts and select Show All Fonts. You can
also use the program \verb+pdffonts+ which comes with \verb+xpdf+ and is
available out-of-the-box on most Linux machines.

\item The IEEE has recommendations for generating PDF files whose fonts
are also acceptable for NIPS. Please see
\url{http://www.emfield.org/icuwb2010/downloads/IEEE-PDF-SpecV32.pdf}

\item LaTeX users:

\begin{itemize}

\item Consider directly generating PDF files using \verb+pdflatex+
(especially if you are a MiKTeX user). 
PDF figures must be substituted for EPS figures, however.

\item Otherwise, please generate your PostScript and PDF files with the following commands:
\begin{verbatim} 
dvips mypaper.dvi -t letter -Ppdf -G0 -o mypaper.ps
ps2pdf mypaper.ps mypaper.pdf
\end{verbatim}

Check that the PDF files only contains Type 1 fonts. 
%For the final version, please send us both the Postscript file and
%the PDF file. 

\item xfig "patterned" shapes are implemented with 
bitmap fonts.  Use "solid" shapes instead. 
\item The \verb+\bbold+ package almost always uses bitmap
fonts.  You can try the equivalent AMS Fonts with command
\begin{verbatim}
\usepackage[psamsfonts]{amssymb}
\end{verbatim}
 or use the following workaround for reals, natural and complex: 
\begin{verbatim}
\newcommand{\RR}{I\!\!R} %real numbers
\newcommand{\Nat}{I\!\!N} %natural numbers 
\newcommand{\CC}{I\!\!\!\!C} %complex numbers
\end{verbatim}

\item Sometimes the problematic fonts are used in figures
included in LaTeX files. The ghostscript program \verb+eps2eps+ is the simplest
way to clean such figures. For black and white figures, slightly better
results can be achieved with program \verb+potrace+.
\end{itemize}
\item MSWord and Windows users (via PDF file):
\begin{itemize}
\item Install the Microsoft Save as PDF Office 2007 Add-in from
\url{http://www.microsoft.com/downloads/details.aspx?displaylang=en\&familyid=4d951911-3e7e-4ae6-b059-a2e79ed87041}
\item Select ``Save or Publish to PDF'' from the Office or File menu
\end{itemize}
\item MSWord and Mac OS X users (via PDF file):
\begin{itemize}
\item From the print menu, click the PDF drop-down box, and select ``Save
as PDF...''
\end{itemize}
\item MSWord and Windows users (via PS file):
\begin{itemize}
\item To create a new printer
on your computer, install the AdobePS printer driver and the Adobe Distiller PPD file from
\url{http://www.adobe.com/support/downloads/detail.jsp?ftpID=204} {\it Note:} You must reboot your PC after installing the
AdobePS driver for it to take effect.
\item To produce the ps file, select ``Print'' from the MS app, choose
the installed AdobePS printer, click on ``Properties'', click on ``Advanced.''
\item Set ``TrueType Font'' to be ``Download as Softfont''
\item Open the ``PostScript Options'' folder
\item Select ``PostScript Output Option'' to be ``Optimize for Portability''
\item Select ``TrueType Font Download Option'' to be ``Outline''
\item Select ``Send PostScript Error Handler'' to be ``No''
\item Click ``OK'' three times, print your file.
\item Now, use Adobe Acrobat Distiller or ps2pdf to create a PDF file from
the PS file. In Acrobat, check the option ``Embed all fonts'' if
applicable.
\end{itemize}

\end{itemize}
If your file contains Type 3 fonts or non embedded TrueType fonts, we will
ask you to fix it. 

\subsection{Margins in LaTeX}
 
Most of the margin problems come from figures positioned by hand using
\verb+\special+ or other commands. We suggest using the command
\verb+\includegraphics+
from the graphicx package. Always specify the figure width as a multiple of
the line width as in the example below using .eps graphics
\begin{verbatim}
   \usepackage[dvips]{graphicx} ... 
   \includegraphics[width=0.8\linewidth]{myfile.eps} 
\end{verbatim}
or % Apr 2009 addition
\begin{verbatim}
   \usepackage[pdftex]{graphicx} ... 
   \includegraphics[width=0.8\linewidth]{myfile.pdf} 
\end{verbatim}
for .pdf graphics. 
See section 4.4 in the graphics bundle documentation (\url{http://www.ctan.org/tex-archive/macros/latex/required/graphics/grfguide.ps}) 
 
A number of width problems arise when LaTeX cannot properly hyphenate a
line. Please give LaTeX hyphenation hints using the \verb+\-+ command.


\subsubsection*{Acknowledgments}

Use unnumbered third level headings for the acknowledgments. All
acknowledgments go at the end of the paper. Do not include 
acknowledgments in the anonymized submission, only in the 
final paper. 

\subsubsection*{References}

References follow the acknowledgments. Use unnumbered third level heading for
the references. Any choice of citation style is acceptable as long as you are
consistent. It is permissible to reduce the font size to `small' (9-point) 
when listing the references. {\bf Remember that this year you can use
a ninth page as long as it contains \emph{only} cited references.}

\small{
[1] Alexander, J.A. \& Mozer, M.C. (1995) Template-based algorithms
for connectionist rule extraction. In G. Tesauro, D. S. Touretzky
and T.K. Leen (eds.), {\it Advances in Neural Information Processing
Systems 7}, pp. 609-616. Cambridge, MA: MIT Press.

[2] Bower, J.M. \& Beeman, D. (1995) {\it The Book of GENESIS: Exploring
Realistic Neural Models with the GEneral NEural SImulation System.}
New York: TELOS/Springer-Verlag.

[3] Hasselmo, M.E., Schnell, E. \& Barkai, E. (1995) Dynamics of learning
and recall at excitatory recurrent synapses and cholinergic modulation
in rat hippocampal region CA3. {\it Journal of Neuroscience}
{\bf 15}(7):5249-5262.
}
\end{comment}

\end{document}
