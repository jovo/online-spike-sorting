
\begin{center}
\begin{figure}[h!]
\begin{subfigure}[b]{.3\textwidth}
\includegraphics[width=\textwidth]{../figs/new/pcaelements.pdf}
\caption{}
\label{fig:ICold}
\end{subfigure}
\begin{subfigure}[b]{.3\textwidth}
% \includegraphics[width=\textwidth]{../figs/new/ICclusteroldpca.pdf}
\caption{}
\label{fig:ICold}
\end{subfigure}
\begin{subfigure}[b]{.3\textwidth}
% \includegraphics[width=\textwidth]{../figs/new/ICclusteroldpca.pdf}
\caption{}
\label{fig:ICold}
\end{subfigure}
\caption{\jovo{dictionary: (a) from first 5 secs, (b) from all data, (c) spectrum from all data which is cumsum/sum} 
} \label{fig:timing}
\end{figure}
\end{center}

\begin{figure}[htbp]
	\centering
		% \includegraphics[height=3in]{../figs/new/pairs.pdf}
	\caption{(a) This shows the average number of true positives versus the average number of false positives in the intracellular cluster for 2 minute segments of the 4 minutes of the experiment.  \smug\ does better than    \jovo{let's make the symbols different for the different methods.  also, let's make the axes in terms of percentages, rather than raw numbers.} }
	\label{fig:asdf}
\end{figure}


\begin{figure}[htbp]
	\centering
		\includegraphics[height=3in]{../figs/new/pairs.pdf}
	\caption{caption}
	\label{fig:pairs}
\end{figure}



\begin{center}
\begin{figure}[h!]
\begin{subfigure}[b]{.2\textwidth}
\includegraphics[width=1\textwidth]{../figs/3dev}
\caption{}
\label{3dev}
\end{subfigure}
% \begin{subfigure}[b]{.28\textwidth}
% \includegraphics[width=\textwidth]{../figs/3devim/clus1}
% \caption{}
% \label{ex31}
% \end{subfigure}
% \begin{subfigure}[b]{.28\textwidth}
% \includegraphics[width=\textwidth]{../figs/3devim/clus2}
% \caption{}
% \label{ex32}
% \end{subfigure}
\begin{subfigure}[b]{.5\textwidth}
\includegraphics[width=\textwidth]{../figs/new/3chpca}
\caption{}
\label{3chpca}
\end{subfigure}
\caption{
Improving \smug\ by incorporating \emph{multiple} channels.
(a) Three electrode device showing local proximity of electrodes with channel indexes in large, red numbers. 
% (b,c) Top 2 most prevalent waveforms.  Each waveform shape is 2ms long.   Note in (a) we have a waveform that appears on both channel 2 and channel 3, whereas in (b) the waveform only appears in channel 3.  If only channel 3 was used, it would be difficult to separate the waveform in (a) and (b), as is demonstrated in Fig.\ 
(b) The representation of detected spikes on the 3rd channel in PCA space. This cluster does not seem separable here.
}
\end{figure}
\end{center}



\begin{figure}[htbp]
	\centering
		% \includegraphics[height=3in]{../figs/new/pairs.pdf}
	\caption{(a) intracellular waveform shape over time (b) and in PC space.}
	\label{fig:pairs}
\end{figure}



\begin{center}
\begin{figure}
	% \includegraphics[width=.5\textwidth]{../figs/IntracellularTrueFalsePositivesv2}
	% \includegraphics[width=.5\textwidth]{../figs/new/ICclusteroldpca.pdf}
		\begin{subfigure}[b]{.5\textwidth}
	\includegraphics[width=\textwidth]{../figs/IntracellularTrueFalsePositivesv2}
	\caption{}
	\label{truewaveforms}
	\end{subfigure}
\begin{subfigure}[b]{.5\textwidth}
\includegraphics[width=\textwidth]{../figs/new/ICclusteroldpca.pdf}
\caption{}
\label{fig:ICold}
\end{subfigure}
% \begin{subfigure}[b]{.33\textwidth}
% \includegraphics[width=\textwidth]{../figs/new/ICclusternewpca.pdf}
% \caption{}
% \label{fig:ICnew}
% \end{subfigure}
\caption{False and true positive detections have the same first-order statistics, making detection using only these statistics quite difficult.  (a)
 Errorbar plots of the true positives, false positives, and missed positives  in the IC cluster.  While the false positives have slightly more variability, the mean shape for the false positives and the true positives is nearly identical.  The true misses have a significantly lower amplitude as well as high variability. (b) All waveforms from the IC neuron as well as those we estimated from the IC neuron projected onto the first two PC space.} \label{fig:IC-PCA}
\end{figure}
\end{center}



\begin{center}
\begin{figure}
\begin{subfigure}[b]{.12\textwidth}
\includegraphics[width=0.8\textwidth]{../figs/8dev}
\caption{}
\label{8dev}
\end{subfigure}
\begin{subfigure}[b]{.28\textwidth}
\includegraphics[width=\textwidth]{../figs/8devim/clus3}
\caption{}
\label{ex81}
\end{subfigure}
\begin{subfigure}[b]{.28\textwidth}
\includegraphics[width=\textwidth]{../figs/8devim/clus9}
\caption{}
\label{ex82}
\end{subfigure}
\begin{subfigure}[b]{.28\textwidth}
\includegraphics[width=\textwidth]{../figs/8devim/clus6}
\caption{}
\label{ex83}
\end{subfigure}
\caption{
\smug\ multielectrode performance.
(a) 8 electrode device showing local proximity of electrodes with channel indexes in large, red numbers. (b,c,d) Top three most prevalent waveforms.  Each waveform shape is 2 ms long.
} \label{sfig:8}
\end{figure}
\end{center}



\begin{center}
\begin{figure}[h!]
\begin{subfigure}[b]{.5\textwidth}
\includegraphics[width=\textwidth]{../figs/new/timingsinglechannel.pdf}
\caption{}
\label{fig:ICold}
\end{subfigure}
\begin{subfigure}[b]{.5\textwidth}
% \includegraphics[width=\textwidth]{../figs/new/ICclusteroldpca.pdf}
\caption{}
\label{fig:ICold}
\end{subfigure}
\caption{\jovo{timing plots: (a) line (can we put a line on there for 4 channels too?), (b) notched boxplots. } 
} \label{fig:timing}
\end{figure}
\end{center}

