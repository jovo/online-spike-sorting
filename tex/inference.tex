\newcommand{\tx}{\tilde{x}}


We now address the problem of posterior inference over the latent variables given the matrix $\bX$ of multielectrode recordings. Unsurprisingly, exact 
inference is intractable, and we have to resort to approximating the posterior distribution.
There exists a vast literature on approximate inference for Bayesian nonparametric models, especially so for models based on the Dirichlet process.
Traditional approaches are sampling-based, typically involving Markov chain Monte Carlo techniques (see eg.\ \citep{Nea2000, IshJam2001}), 
and recently there has also been work on constructing deterministic approximations to the intractable posterior (eg.\ \citep{BleJor2006, MinGha2003}).
Our problem is further complicated by two additional factors. The first is the convolutional nature of our observation process, 
where at each time,
we observe a function of the previous observations drawn from the DP mixture model. This is in contrast to the usual situation where one directly observes 
the DPMM outputs themselves.
The second complication is a computational requirement: typical inference schemes are batch methods that are slow and computationally expensive. 
Our ultimate goal, on the other hand, is to perform inference in real time, making these approaches are unsuitable for our purposes.

Keeping the latter objective in mind, we develop an online algorithm for posterior inference. Our algorithm is based on 
\citep{WangDun2009}, though that work was concerned with the usual case of i.i.d.\ observations from a DPMM. We generalize this algorithm to our 
observation process, and also to account for the time-evolution of the cluster parameters.

At a high level, at time $t$, our algorithm maintains the set of times of the spikes it has inferred from the observations until time $t$. It also maintains
the identities of the neurons that it assigned each of these spikes to, as well as the weight vectors determining the shapes of the associated spike 
waveforms. In addition to these point estimates, the algorithm also keeps a set of posterior distributions $q_{it}(\theta^*_i)$ where $i$ spans over the
set of neurons associated with the spikes seen so far. Let the number of unique neurons observed at time $t$ be $C_t$, so that $i$ varies from $1$ to $C_t$.
For each $i$, $q_{it}(\theta^*_i)$ approximates the distribution over the parameters 
$\theta_i^* \equiv (\mu_i^*, \Sigma_i^*)$ of neuron $i$ given the observations until time $t$. 

Having identified the location and shape of spikes from earlier times, we can calculate their contribution to the recording $x_{t+1}$ at time $t+1$.
We subtract this term from $x_{t+1}$, and treat the residual $\mt{\bx}_{t+1}$ as an observation from a DP mixture model.
Given this residual, our algorithm then makes a hard decision about whether or not this was produced by an underlying spike, what neuron that spike belongs 
to (one of the earlier neurons or a new neuron), and what the shape of the associated spike waveform is. The latter is used to calculate
$q_{i,t+1}(\theta^*_i)$, the new distribution over neuron parameters at time $t+1$. Our algorithm proceeds recursively in this manner. 


Below, we describe each of these steps for the case a single electrode; %Most steps are repeated across all electrodes; 
%we will point out when the need for anything more complicated arises. 
generalizing to the multielectrode case is straightforward. We start by recalling that the basis functions $\bD$, and thus all spike waveforms,
span $L$ time bins. 
Let $z_t$ indicate whether or not a spike is present at time $t$, with $\by_t$ giving its shape, and $\nu_t$ its associated neuron. 
The residual at time $t$ is then given by
\begin{align}
  \mt{\bx}_t = \bx_t - \sum_{i=1}^L \bD\by_{t-i} \label{eq:resid}
\end{align}
Given this residual, the first step is to decide whether or not there is a spike underlying this residual.
By Bayes' rule,
\begin{align}
  P(z_t = 1 | \mt{\bx}_t)  &\propto P(z_t = 1,  \mt{\bx}_t) = \sum_{\nu_t = 1}^{C_{t-1}+1} P(\mt{\bx}_t, {\nu_t} | z_t = 1) P(z_t = 1) \label{eq:spk_prob}\\
\intertext{Recall that $C_{t-1}$ is the number of unique neurons underlying the observations before time $t$. Letting $n_i$ be the number of spikes from 
neuron $i$,  $P({\nu_t} = i | z_t = 1)$ follows from the CRP update rule (equation \eqref{eq:crp_marg_pr}). On the other hand,}
  P(\mt{\bx}_t | {\nu_t = i} , z_t = 1) &= \int_{\Theta} P(\mt{\bx}_t | \theta_t) q_{it} (\theta_t) \dd \theta_t  \label{eq:norm_nw}
\end{align}
Here,  $P(\mt{\bx}_t | \theta_t)$ is just the normal distribution with mean $\mu_t$ and variance $\theta_t$, while we restrict $q_{it}(\cdot)$ be the 
normal-Wishart distribution. % with parameters 
We can then evaluate integral \eqref{eq:norm_nw}, and then summation \eqref{eq:spk_prob} to calculate $P(z_t = 1 | \mt{\bx}_t)$. 
If this exceeds a threshold of $0.5$ we decide that there is a spike present at time $t$, otherwise, we set $z_t = 0$.
Observe that making this decision involves marginalizing over all possible cluster assignments $\nu_t$, and all values of the weight vector $\by_t$.
On the other hand, having decided that a spike is present , we simplify matters collapsing these posterior distributions to point estimates. In
particular, we assign the spike to the neuron with highest posterior probability, and similarly set the weight-vector to its MAP value. 

Given these point estimates, we can update the posterior distribution over parameters of cluster $\nu_t$ to obtain $q_{i,t+1}$ from $q_{i,t}$; this
is straightforward because of conjugacy. 
We need to follow this up with an additional update step for the parameter distributions of \emph{all} clusters because of the AR process on the parameters.
This too is straightforward {\color{red} Get exact update rule}.

The online algorithms were all run with weakly informative parameters (\dec{add parameters once I get vinayak's notation}). The parameters were insensitive to minor changes.  Running time in unoptimized MATLAB code for 4 minutes of data was 31s was a single channel and 3 minutes for all 4 channels on a 3.2 GHz Intel Core i5 machine with 6 GB of memory.
