
Our overall model is then:
\begin{subequations}
\begin{align}
  \mc{T}_i\ \  &\sim \text{PP}(\lambda_i) \quad i \in \NN, \quad &\text{ where } \mathsf{\Lambda}(\cdot)&=\sum_{i=1}^{\infty} \lambda_i \delta_{\theta^*_i} \sim \Gamma \text{P}(\alpha, \mathsf{H}(\cdot| {\phi})), \\ %\mathcal{NW}(\mu, \Sigma)) \\ \\
\vspace{-.8in}
  x_i(t) &= \sum_{j = 1}^{|\mc{T}_i|}  \sum_{k = 1}^{K} y^*_{ijk} \mathsf{d}_k(t - \tau_{ij}), \quad &\text{ where }\by^*_{ij}  &\sim \mathsf{N}_K(\mb{\mu}^*_i, \Sigma^*_i) \quad i,j \in \NN, \\
  x(t)   &= \sum_{i=1}^{\infty} x_i(t) + \eps_t, \quad &\text{ where at any time $t$, } \eps_t &\sim \mathsf{N}(0,\Sigma_x) \text{ independently}
\end{align}
\end{subequations}

\begin{algorithm}
\caption{Generative mechanism for the multi-electrode, non-stationary, discrete-time process}\label{alg:gen_proc}
\begin{tabular}{p{1.2cm}p{12.4cm}}
Input:&  a) the number of bins $T$, and the bin-width $\Delta$\\
  &  b) the $K$-by-$L$ dictionary $\bD$ of $K$ basis functions\\
  &  c) the DP hyperparameters $\alpha$ and $\phi$.\\ 
  &  d) the transition matrix $\bB$ of the neuron AR process \\
Output:& \  An $M$-by-$T$ matrix $\bX$ of multielectrode recordings. % defined by a set of state and time pairs.
\end{tabular}
\begin{algorithmic}[1]
\State Initialize the number of clusters $C_1$ to $0$.
\State Draw the overall spiking rate $\Lambda \sim \text{Gamma}(\alpha, 1)$.
%\State Set $A_{s_i}(\tau) = \sum_j A_{s_i,j} (\tau)$ and define $u_{s_i}(\tau) \ge A_{s_i}(\tau) \forall \tau$. \label{alg:loop}
%\State Let $\tau_o = (w_{i} - l_i)$. \label{alg:smjp_loop}
\For{$t$ in $[T] $}
\State Sample $\mt{z}_t \sim \text{Bernoulli}(\Lambda \Delta)$, with $\mt{z}_t = 1$ indicating a spike in bin $t$.
\If{$\mt{z}_t = 1$}   \label{enum:thin}
  \State Sample $\mt{\nu}_t$, assigning the spike to a neuron, with
%\begin{align*}
$  \mathsf{P}({\mt{\nu}_t} = i) \propto 
  \begin{cases}
   |\mc{T}^t_i| \quad i \in [C] \\
   \alpha \quad\ i = C + 1 \\
  \end{cases}$
%\end{align*}
       \If{ $\nu_t = C_t + 1$} 
          \State  $C_{t+1} \leftarrow C_t + 1$. 
		\State Set $\theta^*_{C_{t+1}} \sim H_{\phi}(\cdot)$, and $\mc{T}_{C_{t+1}}$=\{t\}.
       \Else \State  $\mc{T}_{\nu_t} \leftarrow \mc{T}_{\nu_t} \cup \{t\}$.
    \EndIf
\State Set $\theta_t = \theta^*_{\nu_t}$, recalling that $\theta_t \equiv (\mb{\mu}_t, \Sigma_t)$.
\State Sample $\by_t = (\by^1_t; \cdots; \by^M_1) %\equiv (\by_{t1}, \cdots, \by_{t\Upsilon}) 
           \sim \mathsf{N}(\mb{\mu}_t, \Sigma_t)$, determining the spike shape at all electrodes.
%\State $\bx^h_{t:t+L} = A\by^h$
\State $ x^m_t = \sum_{h = 1}^L \bD_{:,h}^{\T} \by^m_{t-h-1} + \epsilon^m_t \text{\qquad where $\epsilon^m_t \sim \mathsf{N}(0,\sigma^2), m \in [M]$.} $
\State Update the cluster parameters: ${\mb{\mu}}^*_i = \mathbf{B} {\mb{\mu}}^*_i + r_i \quad i \in [C_{t+1}]$
\EndIf
\EndFor
\end{algorithmic}
\end{algorithm}

