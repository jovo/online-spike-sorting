We perform inference in an online manner \cite{WangDun2009}. As observations arrive, our inference algorithm decides whether or not a 
new spike is present, which neuron (cluster) to assign that spike to, as well as the shape of the spike waveform. On the other hand, our algorithm 
maintains a posterior distribution over the cluster parameters that characterize the distribution over shapes. Having identified the location and shape of 
spikes from earlier times, we subtract these from the observations treat the residual as an observation from a DP mixture model.
The cluster assignment of earlier spikes determines the seating arrangement of customers in the Chinese restaurant associated with the DP. Given the
corresponding distribution over parameters, $p(\theta)$, we decide whether there is an underlying spike, which cluster it is assigned to, and what
the shape of that spike is. We simultaneously update the distribution over parameters of clusters. Assume each spike waveform spans $W$ time intervals. 
Define the residual at time $t$ as $X_t - \sum_{i=1}^W A$. At time $t$, let $y_t$ represent the shape of the action potential.
Letting $\tilde{x}_t$ be the observation at time $t$, we have
\begin{align}
  z_t  & \sim Bern(p) \\
  \intertext{\hspace{2in} if $z_t == 1$}
  \gamma_t| \gamma_{1:t-1} &\sim CRP \\
  \theta_t| \gamma_t = i & \sim \mathcal{N}(\theta_{t-1},\Sigma)
\end{align}

\bibliography{refs}
\bibliographystyle{unsrt}