\newcommand{\tx}{\tilde{x}}
\newcommand{\resx}{\delta{\bx}^L}


We now address the problem of posterior inference over the latent variables given the matrix $\bX$ of multielectrode recordings. 
%Unsurprisingly, exact inference is intractable, and we have to resort to approximating the posterior distribution.
There exists a vast literature on computational approaches to posterior inference for Bayesian nonparametric models, especially so for models based on the 
Dirichlet process.
Traditional approaches are sampling-based, typically involving Markov chain Monte Carlo techniques (see eg.\ \citep{Nea2000, IshJam2001}), 
and recently there has also been work on constructing deterministic approximations to the intractable posterior (eg.\ \citep{BleJor2006, MinGha2003}).
Our problem is complicated by two additional factors. The first is the convolutional nature of our observation process, 
where at each time,
we observe a function of the previous observations drawn from the DPMM. This is in contrast to the usual situation where one directly observes 
the DPMM outputs themselves.
The second complication is a computational requirement: typical inference schemes are batch methods that are slow and computationally expensive. 
Our ultimate goal, on the other hand, is to perform inference in real time, making these approaches are unsuitable for our purposes.

Keeping the latter objective in mind, we develop an online algorithm for posterior inference. Our algorithm is based on the sequential update and
greedy search (SUGS) algorithm of
\citep{WangDun2009}, though that work was concerned with the usual case of i.i.d.\ observations from a DPMM. We generalize SUGS to our 
observation process, also accounting for the time-evolution of the cluster parameters.

Below, we describe an iteration of our algorithm for the case a single electrode; %Most steps are repeated across all electrodes; 
%we will point out when the need for anything more complicated arises. 
generalizing to the multielectrode case is straightforward. 
At time $t$, our algorithm maintains the set of times of the spikes it has inferred from the observations so far. It also maintains
the identities of the neurons that it assigned each of these spikes to, as well as the weight vectors determining the shapes of the associated spike 
waveforms. We indicate these point estimates with the hat operator, so, for example $\mh{\mc{T}}^t_i$ is the set of estimated spike times before time $t$ assigned
to neuron $i$. The algorithm also keeps a set of posterior distributions $\mathsf{q}_{it}(\theta^*_i)$ where $i$ spans over the
set of neurons associated with the spikes seen so far (i.e.\ $i \in [\mh{C}_t]$). 
For each $i$, $\mathsf{q}_{it}(\theta^*_i)$ approximates the distribution over the parameters 
$\theta_i^* \equiv (\mu_i^*, \Sigma_i^*)$ of neuron $i$ given the observations until time $t$. 

Having identified the time and shape of spikes from earlier times, we can calculate their contribution to the recordings 
$\bx^L_t \equiv (x_{t},\cdots, x_{t+L-1})\T$.
Recalling that the basis functions $\bD$, and thus all spike waveforms, span $L$ time bins, the residual at time $t+t_1$ is then given by
% \begin{align}
 $ \delta x_{t+t_1} = x_t - \sum_{h=1}^{L-t_1} \mh{z}_{t-h} \bD\mh{\by}_{t-h}$. %\label{eq:resid}
% \end{align}
% 
We treat the residual $\resx_{t} =  (\delta x_{t}, \cdots, \delta x_{t+L})\T$ as an observation from a DP mixture model, and use this to make a hard decision about whether or not this was produced 
by an underlying spike, what neuron that spike belongs 
to (one of the earlier neurons or a new neuron), and what the shape of the associated spike waveform is. The latter is used to calculate
$\mathsf{q}_{i,t+1}(\theta^*_i)$, the new distribution over neuron parameters at time $t+1$. Our algorithm proceeds recursively in this manner. 
% 
% 
%Let $z_t$ indicate whether or not a spike is present at time $t$, with $\by_t$ giving its shape, and $\nu_t$ its associated neuron. 
For the first step we use Bayes' rule to decide whether there is a spike underlying the residual:\vspace{-5pt}
\begin{align}
  \mathsf{P}(\mt{z}_t = 1 | \resx_t)  &\propto \mathsf{P}(\mt{z}_t = 1,  \resx_t) = \sum_{\nu_t \in \mh{C}_{t-1}+1} \mathsf{P}(\resx_t, {\nu_t} | \mt{z}_t = 1) \mathsf{P}(\mt{z}_t = 1) \label{eq:spk_prob}\\
  \mathsf{P}(\resx_t | {\nu_t = i} , \mt{z}_t = 1) &= \int_{\Theta} \mathsf{P}(\resx_t | \theta_t) \mathsf{q}_{it} (\theta_t) \dd \theta_t  \label{eq:norm_nw}
\end{align}
where $\mathsf{P}({\nu_t} = i | \mt{z}_t = 1)$ follows from the $\mathsf{CRP}$ update rule (Eq.\ \eqref{eq:crp_marg_pr}).
Here,  $\mathsf{P}(\resx_t | \theta_t)$ is just the normal distribution, while we restrict $\mathsf{q}_{it}(\cdot)$ be the 
normal-Wishart distribution. % with parameters 
We can then evaluate integral \eqref{eq:norm_nw}, and then summation \eqref{eq:spk_prob} to approximate $\mathsf{P}(\mt{z}_t = 1 | \resx_t)$. 
If this exceeds a threshold of $0.5$ we decide that $\mt{z}_t=1$; otherwise,  $\mt{z}_t = 0$.
Observe that making this decision involves marginalizing over all possible cluster assignments $\nu_t$, and all values of the weight vector $\by_t$.
On the other hand, having decided that a spike is present, we simplify matters by collapsing these posterior distributions to point estimates 
$\mh{\nu}_t$ and $\mh{\by}_t$. Both are obtained from their MAP values. 

Given these point estimates, we can update the posterior distribution over parameters of cluster $\mh{\nu}_t$ to obtain $\mathsf{q}_{i,t+1}(\cdot)$ from 
$\mathsf{q}_{i,t}(\cdot)$; this is straightforward because of conjugacy. We follow this up with an additional update step for the distributions of the means of
\emph{all} clusters: this is to account for the AR evolution of the cluster means. 
%This too is straightforward {\color{red} Get exact update rule. Hyperparameters}.
We use a variational update to keep  $\mathsf{q}_{i,t+1}(\cdot)$ in the normal-Wishart distribution. Finally we take a stochastic gradient step to
update any hyperparameters we wish to learn. We provide all details in the Supplementary material.

%The online algorithms were all run with weakly informative parameters (\dec{add parameters once I get vinayak's notation}). The parameters were insensitive to minor changes.  Running time in unoptimized MATLAB code for 4 minutes of data was 31s was a single channel and 3 minutes for all 4 channels on a 3.2 GHz Intel Core i5 machine with 6 GB of memory.

