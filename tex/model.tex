\newcommand{\bY}{\mathbf{Y}}
\newcommand{\NN}{\mathbb{N}}         
%\nipsfinalcopy % Uncomment for camera-ready version

\subsection{Notation}

Unless otherwise specified, we let lower-case English alphabet characters indicate scalars $x \in \Real$, bold indicates column vectors $\mb{x} \in \Real^p$, and upper-case bold indicates matrices, $\bX \in \Real^{p \times q}$.  Parameters and constants will be Greek characters.  Time will be $t \in (0,T)$, $i \in [N]$ will index the $N$ neurons, where $[N]=\{1,2,\ldots,N\}$. Script will denote sets and pipes will denote the cardinality of the set, e.g. $|\mc{T}|$.  
\vspace{-.1in}
\subsection{Input}
\vspace{-.1in}
% \subsection{Data Model}
Our data is a time-series of multielectrode recordings $\bX \equiv (\bx_1, \cdots, \bx_T)$, and consists of $T$ recordings from $M$ channels. 
The set of recording times lie on regular grid with interval length $\Delta$, while $\bx_t \in \mathbb{R}^M$ for all $t$. 
This time-series of electrical activity is driven by an unknown number of neurons, and 
we let the number be unbounded. %, though only a few of the infinite
%neurons dominate. These neurons contribute the majority of the activity in any finite interval of time; however, as time passes, the total number of 
%observed neurons increases {\color{red}(Justify?)}. 
%Each neuron, has its own `shape' A natural model in such a situation is to
The neurons themselves generate continuous-time voltage traces, with the outputs of all neurons superimposed and discretely sampled to produce the 
recordings $\bX$.  At a high level, we model the output of each neuron as a
series of idealized spikes which are smoothed with appropriate kernels, the latter determining the shape of each spike. 
%Each neuron has its own distribution over waveform shapes. 
We describe this in detail, starting first with a model for a single channel recording $\bx\T \equiv (x_1, \cdots, x_T)$.

\vspace{-.1in}
\subsection{Modeling a single electrode recording}
\vspace{-.1in}

There is a rich literature characterizing the spiking activity of a single neuron \citep{?}, accounting in detail for factors like non-stationarity, 
refractoriness and spike waveform. We however make a number of simplifying assumptions (some of which we later relax, others we leave
for future work). Figure \ref{fig:schematic} provides a schematic depiction of our generative process.
First, we model the spiking activity of each neuron as stationary and memoryless, so that its set of spike times are 
distributed as a homogeneous Poisson process (PP). 
{\color{red} justify?} We model the neurons themselves are heterogeneous, with the $i^{th}$ neuron having
an (unknown) firing rate $\lambda_i$. Call the ordered set of spike times of the $i^{th}$ neuron $\mc{T}_i$; then the time between successive elements of $\mc{T}_i$ is 
exponentially distributed with mean $1/\lambda_i$. We write this as
\vspace{-.1in}
\begin{align}
  \mc{T}_i &\sim \text{PP}(\lambda_i)
\end{align}
The actual electrical output of a neuron is not binary; instead each spiking event is a smooth perturbation in voltage about a
resting state. This perturbation forms the shape of the spike, and without any loss of generality, we set the resting state to zero. 
{(\color{red} figure? better biological description? comment on how we preprocess the data to get zero mean?)}. 
While the spike shapes vary across neurons as well as across different spikes of the same neuron, each 
neuron has its own characteristic distribution over shapes. 
We let $\bth^*_i \in \Theta$ parametrize this distribution for neuron $i$.
 % having parameter $\bth^*_i$. 
Whenever this neuron emits a 
spike, a new shape is drawn independently from the corresponding distribution. %$p_{\bth_i}$, and 
This waveform is then offset to the time of the spike, and contributes to the voltage trace associated with that spike. The complete detected output of the neuron is the 
superposition of all these spike waveforms plus noise.  
We assume the noise at each observation time is i.i.d.\ Gaussian.

% Comment on how this dictionary is obtained now, or in section on inference?)}. 
We model the random spike shapes themselves as weighted superpositions of a dictionary of $K$ basis functions $\bd(t) \equiv (d_1(t), \cdots, d_K(t))\T$. The
dictionary elements are shared across all neurons, and each is a real-valued function of time, e.g., $d_k \in L_2$.
For the $i^{th}$ neuron, the $j^{th}$ spike $\tau_{ij} \in \mc{T}_i$, is associated with a random $K$-dimensional weight vector $\tby_{ij} \equiv (\ty_{ij1}, \ldots \ty_{ijK})\T$, and the 
shape of this spike at time $t$ is given by the weighted sum $\sum_{k=1}^K \ty_{ijk} d_k(t)$. We assume $\tby_{ij} \sim \mc{N}_K(\mb{\mu}^*_i, \Sigma^*_i)$, indicating a $K$-dimensional 
Gaussian distribution with mean and covariance given by $\theta^*_i \equiv (\mb{\mu}^*_i, \Sigma^*_i)$.   Then, at any time $t$, the output of neuron $i$ is
\begin{align}
  x_{i}(t) &= \sum_{j=1}^{|\mc{T}_i|} \sum_{k=1}^K \ty_{ijk} d_k(t - \tau_{ij}).
\end{align}
% 
% 
% 
% \begin{center}
% \begin{figure}
% % \includegraphics[width=\textwidth]{../figs/truefalsepositive}
% \caption{Schematic of our Generative Model.}
% \label{fig:schmetic}
% \end{figure}
% \end{center}
% 
{The total signal recorded from any electrode  is the superposition of the outputs of all neurons. Assume for the moment there are $N$
neurons, and define $\mc{T} = \cup_{i \in [N]} \mc{T}_i$ as
the (ordered) union of the spike times of all neurons. 
Let $\tau_l \in \mc{T}$ indicate the time of $l^{th}$ overall spike, whereas $\tau_{ij} \in \mc{T}_i$ is the $j^{th}$ spike of neuron $i$.
%To map elements  $\tau_{ij} \in \mc{T}_i$ to elements  $\tau_l \in \mc{T}$,   
This defines a pair of mappings: $\nu:[\mc{T}]\rightarrow [N]$, and $p:[\mc{T}]\rightarrow \mc{T}_{\nu(i)}$, with %$(i = \nu(l))$, , which maps from the $j^{th}$ spikes of neuron $i$ to the $l^{th}$ overall spike.
$\tau_l = \tau_{\nu_l p_l}$. 
%\jovo{i don't think the square brackets around $\mc{T}$ are correct, we are mapping from the set of spikes, not the number of spikes, right?} 
In words, $\nu_l \in N$ is the neuron to which the $l^{th}$ element of $\mc{T}$ belongs, 
while $p_l$ indexes this spike in the spike train $\mc{T}_{\nu_l}$.
%\jovo{``position'' is weird to me.  can we say: ``indexes which spike of neuron $\nu_l$'', or something like that?}.
Let $\bth_l \equiv (\mb{\mu}_l, \Sigma_l)$ be the neuron parameter associated with spike $l$, so that $\bth_l = \bth^*_{\nu_l}$. 
Finally, define $\by_l=(y_{l1}, \ldots, y_{lK})\T \equiv \tby_{\nu_j p_j}$ as the weight vector of spike $\tau_l$. Then, we have that}
% \begin{subequations}
\begin{align}
  x(t) &= \sum_{i \in [N]} x_{i}(t) =   \sum_{l \in |\mc{T}|} \sum_{k \in [K]} y_{lk} d_k(t - \tau_{l}), \qquad %\label{eq:spk_sup} \\
% \intertext{where}
  \text{ where } \by_{l}  \sim \mc{N}_K(\mb{\mu}_{l}, \Sigma_{l}). \label{eq:spk}
\end{align}
% \end{subequations}
% 
From the superposition property of the Poisson process \citep{kingman93}, the overall spiking activity $\mc{T}$ is 
Poisson with rate $\Lambda = \sum_{i \in [N]} \lambda_i$. Each event $\tau_l \in \mc{T}$ is associated with a pair of labels, the parameter of the neuron to which it 
is assigned ($\bth_l = (\mb{\mu}_l, \Sigma_l)$), and the weight-vector characterizing the spike shape ($\by_l$). We view these as the ``marks'' of a 
marked Poisson process $\mc{T}$.  From the properties of the Poisson process, we have that the marks $\bth_l$ are drawn i.i.d. from a probability measure 
\vspace{-.08in}
\begin{align}
 G(\dd \bth) = \frac{1}{\Lambda}\sum_{i \in [N]} \lambda_i \delta_{\bth^*_i}    \label{eq:mark_distr}
\end{align}
Note that with probability one, the neurons have distinct parameters, so that the mark $\bth_l$ associated with spike $l$ identifies the
neuron which produced it: $G(\bth_l = \bth^*_i) = P(\nu_l= i) = \lambda_i/\Lambda$. Given $\bth_l$, $\by_l$ is distributed as in
Eq.~\eqref{eq:spk}. The output waveform $x(t)$ is then a linear functional of this marked Poisson process. % (Eq.~\eqref{eq:spk_sup}). 

\vspace{-.1in}
\subsection{Completely random measures (CRMs)}

The previous section assumed a known number of neurons $N$. In practice however, our recordings are a superposition of the outputs of an unknown
number of neurons. We deal with this uncertainty by taking a nonparametric Bayesian approach, and letting $N$ be infinite.
While the number of neurons observed over any finite observation interval is finite, this number increases with the observation interval. 
%This makes sense in a biological context, in that as we record for longer (e.g., days, weeks, or months even), certain neurons will die or drop-out, and others will appear, simply by virtue of the electrodes moving, for example. 
Such an approach leads to an elegant and flexible modeling framework, and has  already proved successful in neuroscience applications
\citep{WoodBla2008}.
Since only a finite number of spikes are observed in any finite interval, the total rate $\Lambda$ must 
also be finite. Moreover, we want this to be dominated by a few $\lambda_i$: the corresponding neurons contribute the majority of the spiking
activity in the observation interval. 
A natural framework that captures these  modeling requirements is that of \emph{completely random measures} (CRMs) \citep{Kingman:PJM67}.
Completely random measures are stochastic processes that form flexible and convenient priors over
infinite dimensional objects like probability distributions \citep{JamesLP09}, hazard functions \citep{Hjo1990}, latent features \citep{ThiJor2007} etc. 
These have been well studied in the Bayesian nonparametrics and machine learning communities, and there exists a wealth of literature on
their theoretical properties, as well as on computational approaches to posterior inference.

Recall that each neuron is characterized by a pair of parameters $(\lambda_i, \bth^*_i)$; the former characterizes the distribution over spike times, 
and the latter, over spike
shapes. With Eq.~\eqref{eq:mark_distr} in mind, we map the infinite collection of pairs $\{(\lambda_i, \bth^*_i)\}$ to an atomic measure on $\Theta$:
\vspace{-.1in}
\begin{align}
  \Lambda(\dd \bth) = \sum_{i=1}^{\infty} \lambda_i \delta_{\bth^*_i}.
\end{align}
For any subset $\varTheta$ of $\Theta$, the measure $\Lambda(\varTheta)$ equals \( \sum_{\{ i: \bth^*_i \in \varTheta \} } \lambda_i\). We allow $\Lambda(\cdot)$ to be random,
modeling it as a realization of a completely random measure. Such a random measure has the property that for any two disjoint subsets $\varTheta_1$,  $\varTheta_2 \subseteq \Theta$, the measures $\Lambda(\varTheta_1)$ and $\Lambda(\varTheta_2)$ are independent. 
This distribution over measures is induced by a distribution
over the infinite sequence of weights (the $\lambda_i$'s), and a distribution over the sequence of their locations (the $\bth^*_i$'s). 
For a CRM, the weights $\lambda_i$ are the jumps of a \Levy process \citep{Sato90}, and their distribution is characterized by a 
\Levy measure $\rho(\lambda)$. The locations $\bth^*_i$ are drawn i.i.d.\  from a base probability measure $H(\bth^*)$.
As is typical, we assume these to be independent (though this is not necessary). 
%{\color{green} if there's space, I
%can elaborate on the construction of the CRM from its Levy measure, though this is not necessary}

The particular class of CRM is determined by the \Levy measure $\rho(\lambda)$. For our application, we set $\rho(\lambda) = \alpha \lambda^{-1}\exp(-\lambda)$;
this results in a CRM called the Gamma process ($\Gamma$P) \citep{applebaum2004}. 
The Gamma process has the convenient property that the 
total rate $\Lambda \equiv \Lambda(\Theta) = \sum_{i=1}^{\infty} \lambda_i$ is Gamma distributed (and thus conjugate to the Poisson process prior on $\mc{T}$).
%\footnote{We abuse notation by using $\Lambda$ to denote both the measure as well as the total measure of $\Theta}
%The Gamma distribution has shape parameter $1$ and scale parameter $\alpha$.  Since this is finite almost surely, so too is $\mc{T}$. 
The Gamma process is also closely connected with the Dirichlet process \citep{Ferguson73}, which will prove useful
later on.
Other choices of the \Levy intensity can be used to capture greater uncertainty in the number of neurons active in any finite interval, or to model
power-law behavior in the number of spikes emitted by different neurons.

To complete the specification on the Gamma process, we choose a base-measure $H(\bth^*)$.
Recalling that $\bth^* \equiv (\mu^*, \Sigma^*)$ gives the mean and variance of the weight-vector $\by^*$ of a neuron, we set $H(\bth^*)$ 
to be the conjugate normal-Wishart distribution with parameters $\phi$. Our overall model is then:
\vspace{-.1in}
\begin{subequations}
\begin{align}
  \mc{T}_i\ \  &\sim \text{PP}(\lambda_i) \quad i \in \NN, \qquad &\text{ where } \Lambda(\cdot)&=\sum_{i=1}^{\infty} \lambda_i \delta_{\theta^*_i} \sim \Gamma \text{P}(\alpha, H(\cdot| {\phi})), \\ %\mathcal{NW}(\mu, \Sigma)) \\ \\
  x_i(t) &= \sum_{j = 1}^{|\mc{T}_i|}  \sum_{k = 1}^{K} y^*_{ijk} d_k(t - \tau_{ij}), \qquad &\text{ where }\by^*_{ij}  &\sim \mc{N}_K(\mb{\mu}^*_i, \Sigma^*_i) \quad i,j \in \NN, \\
  x(t)   &= \sum_{i=1}^{\infty} x_i(t) + \eps_t, \qquad &\text{ where } \eps_t &\sim \mc{N}(0,\Sigma_x)
\end{align}
\end{subequations}
%Each spike of each neuron is associated with a time $e$ and a weight vector $y$, and one can view the model above as a doubly stochastic Poisson
%process on the product space. 
% 
%\jovo{perhaps it is standard, but you sample $\Lambda$ and then the next line you have $\lambda_i$, but no explanation for how you from $\Lambda$ to $\lambda_i$. i also am confused as to why $\phi$ is a subscript on $H$, rather than in the $(\cdot)$}
It will be more convenient to work with the marked Poisson process representation of Eq.~\eqref{eq:spk}. % and \eqref{eq:spk_shape}. 
The superposition process $\mc{T}$ is a rate $\Lambda$ Poisson process,
and under a Gamma process prior, $\Lambda$ has a Gamma distribution with shape and scale parameters $\alpha$ and $1$ respectively \citep{Ferguson73}.
As we saw (Eq.~\eqref{eq:mark_distr}), the labels $\bth_i$ assigning events to neurons are drawn i.i.d. from a normalized Gamma 
process $G(\dd \bth)$:
\vspace{-.2in}
\begin{align}
 G(\dd \bth) = \frac{1}{\Lambda} \sum_{l=1}^{\infty} \lambda_l.
\end{align}
$G(\dd \bth)$ is a random probability measure called a \emph{normalized random measure} \citep{JamesLP09}. Crucially, a 
normalized Gamma process is the Dirichlet process (DP) \citep{Ferguson73}, so that $G$ is a draw from a Dirichlet process. For the $l^{th}$ spike in $\mc{T}$, given its 
parameter $\bth_l$, its shape vector is drawn from a normal distribution
with parameters $(\mb{\mu}_{l}, \Sigma_{l})$. Thus the spike parameters $\bth$ are i.i.d.\ draws from a Dirichlet process, while the weight vectors are
draws from a DP mixture model of Gaussians (DPMM) \citep{Lo1984}.

The connection with the DP allows us to simplify the representation of our model. In particular, we exploit a remarkable property of the DP that
allows us to integrate out the infinite-dimensional variable $G(\cdot)$. The resulting marginal distribution over observations follows the so-called
 Chinese restaurant process (CRP) \citep{Pit2002a}. Under this scheme, the $l^{th}$ spike is assigned the same parameter as an earlier spike with probability 
proportional to the number of earlier spikes having that parameter. It is assigned a new parameter (and thus, a new neuron is observed) with probability 
proportional to $\alpha$. Letting $C_t$ be the number of neurons observed until time $t$, and  $|\mc{T}^t_i| = |\mc{T}_i \cap [0,t)|$ be the number of spikes produced by neuron $i$ before time $t$,
we then have for spike $l$ at time $t = \tau_l$: 
\vspace{-.06in}
\begin{align}
  P({\nu_l} = i) & \propto 
  \begin{cases}
   |\mc{T}^t_i| \quad i \in \{1,\cdots, C_{t}\}, \\
   \alpha \quad\ i = C_{t} + 1, 
  \end{cases}  
\label{eq:crp_marg_pr}
\end{align}
%\footnote{Strictly speaking, the process we just described is called a P\'olya urn scheme \citep{BlaMac1973}; for simplicity, we do not distinguish between this and the CRP.}.
and $\bth_l = \bth^*_{\nu_l}$. 
This marginalization property of the DP allows us to integrate out the infinite-dimensional rate vector $\Lambda(\cdot)$, and sequentially 
assign spikes to neurons based on the assignments of earlier spikes.
%is assigned (or equivalently, the parameter $\bth$ associated with that neuron). These marks are drawn from a probability measure 
%$G(\dd \bth) = \frac{1}{R} R(\dd \bth)$. From the properties of the Gamma process, the probability measure $G$ a Dirichlet process, 
Doing so requires one final property: for the Gamma process, the random probability measure $G(\cdot)$ is independent of the total mass $\Lambda$. 
Consequently, the clustering of spikes (determined by $G(\cdot)$) is independent of the rate $\Lambda$ at which they are produced. We then have
 the following model equivalent to the one above:
\begin{subequations}
\begin{align}
  \mc{T} &\sim \text{PP}(\Lambda), \qquad &\text{ where } \Lambda  &\sim \text{Gamma}(\alpha, 1),
   \\
   (\mb{\mu}_{l}, \Sigma_{l}) &\equiv \bth_l,  &\text{ where } \bth_l &\sim \text{CRP}(\alpha, H_{\phi}(\cdot)), \quad l \in [|\mc{T}|],   \label{eq:CRP}\\
  x(t) &=   \sum_{l=1}^{|\mc{T}|} \sum_{k=1}^K y_{lk} d_k(t - \tau_{l}) + \eps_t  &\text{ where } \by_l &\sim \mc{N}_K(\mb{\mu}_{l}, \Sigma_{l}), \,  l \in [|\mc{T}|], \, \eps_t \sim \mc{N}(0,\sigma^2).   \label{eq:CRP_mix}
\end{align}
\end{subequations}
Unlike most applications which observe the outputs of a CRP, our observation at any time $t$ is a convolution-like function of the CRP outputs of all
earlier times. Consequently, we cannot directly apply standard techniques for posterior inference. In \S \ref{sec:inf}, we develop a novel online 
algorithm for posterior inference; first, we provide a discrete-time approximation to our model.

%For neuron $i$, the sequence of spike times is distributed as a Poisson process with random rate $\lambda_i$.
%Each event $\tau_{ij}$ is associated with a mark or label $y_{ij}$ drawn from a normal distribution (again, with random parameters).
%More broadly, we can view the superposed process $\mc{T}$ as a rate $\Lambda$ Poisson process, with each event having a pair of marks, the neuron identity $i$,
%and weight $y$. From the properties of the Gamma process, the pair form a draw from a Dirichlet process.
%Our data is in a form that makes discrete-time modeling more natural, and an approach now is
%one based on the Beta process-binomial process.

\vspace{-.1in}
\subsection{A discrete-time approximation}
In the previous subsections, we modeled the continuous-time voltage waveform output of a neuron. Our data on the other hand consists of recordings
at a discrete set of times. While it is possible to make inferences about the continuous-time process that underlies these discrete recordings,
in this paper, for simplicity, we restrict ourselves to the discrete case. We thus provide a discrete-time approximation to the model above. 
Our approximation follows easily from the marked Poisson process characterization of the model.

Recall first the Bernoulli approximation to the Poisson process: a sample from a Poisson process with rate $\Lambda$ can be approximated by discretizing
time at a granularity $\Delta$, and assigning each bin an event independently with probability $\Lambda\Delta$. The accuracy of this approximation increases 
as $\Delta$ tends to $0$.
%
%This suggests the following approximation at a time resolution $\Delta$. Draw the random Poisson process rate $\Lambda$ drawn from a Gamma$(1,\alpha)$ 
%distribution. Simultaneously, draw a random probability measure
% $G$ from a Dirichlet process. Assign an event to an interval independently with probability $\Lambda\Delta$, and to each event, assign a random mark drawn 
In our case, we have to approximate the marked Poisson process $\mc{T}$. All this requires additionally is to assign marks $\bth_i$ and $\by_i$ to each event 
in the Bernoulli approximation. Following Eqs.~\eqref{eq:CRP} and \eqref{eq:CRP_mix}, the $\bth_l$'s are distributed according
to a Chinese restaurant process, while each $\by_l$ is drawn from a normal distribution parametrized by the corresponding $\bth_l$. We discretize the 
elements of dictionary $d_k \equiv \{d_k(t)\}_{t \in (0,T)}$ as well, defining a mapping from $L_2$ to $\Real^L$ yielding discrete dictionary elements $\mt{d}_k=(\mt{d}_k[1], \ldots, \mt{d}_k[L])\T$,  so  $\bD \in \Real^{K \times L}$. The shape of the $j^{th}$ spike is now a vector of length $L$, and for a weight vector
$\by$, is given by $\bD \by$.

We can simplify notation a bit for the discrete time model. Let $t$ index time-bins (so that for an observation interval of length $T$, $t \in [T/\Delta]$).
Let the binary variable $z_t$ indicate whether or not a spike in present in time bin $t$ (recall that $z_t \sim \text{Bernoulli}(\Lambda \Delta)$). Let
$\nu_t$ and $\theta_t$ be the neuron and neuron parameter associate with the spike in time bin $t$ (these remain undefined if $z_t = 0$).
Then the output at time $t$, $x_t$ is given by
\begin{align}
  x_t = \sum_{\tau = 1}^L \bD_{\tau}^{\T} \by_{t-\tau} + \epsilon_t &\text{\quad where $\bD_{\tau}$ is column $\tau$ of $\bD$.} 
\end{align}
Algorithm \ref{alg:gen_proc} outlines generative mechanism of the data for the discrete time model.
\subsection{Correlations in time and across electrodes}
We now describe two extensions to the model outlined so far. 
%The first is the inclusion of measurement noise: {\color{red} Biology?}. Let $\eps_t$ be the noise at time $t$, we model this as independent, additive and Gaussian.
%However, rather than modeling the noise as independent across time, we model it as a first-order autoregressive process. This can capture
%effects like the movement of electrodes during the experiment. 
The first relaxes the requirement that the parameters $\theta^*$ of each neuron remain constant, instead allowing the Gaussian mean $\mb{\mu}^*$ evolve
with time.  {\color{green} Justify, eg cells dying}. 
%Recall that for neuron $i$, the associated parameters are ($\mb{\mu}^*_i, \Sigma^*_i$). We keep the covariance $\Sigma^*_i$ fixed, however we model the mean $\mb{\mu}^*_i$ as a first-order autoregressive process. 
In continuous-time, we allow these parameters to evolve according to a Gaussian process (GP) \citep{}, so that our original model now involves a DP mixture 
of Gaussian processes. The means continue to remain Gaussian distributed marginally, their time-evolution is determined by the GP covariance kernel.
We choose a Markov kernel, this results in a first-order autoregressive process for the discrete case. Letting $\mb{B}$ be the transition matrix of this
process, 
and $r$ be an independent and normally distributed innovation term, we have
\begin{align}
  \mb{\mu}^*_{t+1} = \mathbf{B} \mb{\mu}^*_t + \mathbf{r}_t.
\end{align}

Our second extension is to generalize out model from a single electrode to the case of multielectrode recordings. 
We do this allowing each channel to have its own latent weight vector: for channel $m$ at time $t$, call this $\by^m_t$.
Of course, these vectors need to be correlated (since they correspond to the same spike). We do this by modeling the set of
vectors as a matrix-variate normal.
{\color{green} Need to check how we set details of this correlation matrix}.

\begin{algorithm}
\caption{Generative mechanism for the multi-electrode, non-stationary, discrete-time process}\label{alg:gen_proc}
\begin{tabular}{p{1.2cm}p{12.4cm}}
Input:&  a) the number of bins $T$, and the bin-width $\Delta$\\
  &  b) the $K$-by-$L$ dictionary $\bD$ of $K$ basis functions\\
  &  c) the DP hyperparameters $\alpha$ and $\phi$.\\ 
  &  d) the transition matrix $\bB$ of the neuron AR process \\
Output:& \  An $M$-by-$T$ matrix $\bX$ of multielectrode recordings. % defined by a set of state and time pairs.
\end{tabular}
\begin{algorithmic}[1]
\State Initialize the number of clusters $C$ to $0$.
\State Draw the overall spiking rate $\Lambda \sim \text{Gamma}(1,\alpha)$.
%\State Set $A_{s_i}(\tau) = \sum_j A_{s_i,j} (\tau)$ and define $u_{s_i}(\tau) \ge A_{s_i}(\tau) \forall \tau$. \label{alg:loop}
%\State Let $\tau_o = (w_{i} - l_i)$. \label{alg:smjp_loop}
\For{$t$ in $[T] $}
\State Sample $z_t \sim \text{Bernoulli}(\Lambda \Delta)$, with $z_t = 1$ indicating a spike in bin $t$.
\If{$z_t = 1$}   \label{enum:thin}
  \State Sample $\nu_t$, assigning the spike to a neuron, with
%\begin{align*}
$  P({\nu_t} = i) \propto 
  \begin{cases}
   |\mc{T}_i| \quad i \in [C] \\
   \alpha \quad\ i = C + 1 \\
  \end{cases}$
%\end{align*}
       \If{ $\nu_t = C + 1$} 
          \State  $C \leftarrow C + 1$. 
		\State Set $\theta^*_{C} \sim H_{\phi}(\cdot)$, and $n_C$=1.
       \Else \State  $n_{\nu_t} \leftarrow n_{\nu_t}+1$.
    \EndIf
\State Set $\theta_t = \theta^*_{\nu_t}$, and recall that $\theta_t \equiv (\mb{\mu}_t, \Sigma_t)$.
\State Sample $\bY_t %\equiv (\by_{t1}, \cdots, \by_{t\Upsilon}) 
           \sim \mathcal{N}(\mb{\mu}_t, \Sigma \otimes K)$, determining the spike shape at all electrodes. \jovo{define $\otimes$.}
%\State $\bx^h_{t:t+L} = A\by^h$
\EndIf

\State Update the cluster parameters: ${\mb{\mu}}^*_i = \mathbf{B} {\mb{\mu}}^*_i + r_i \quad i \in \{1, \cdots, C\}$
\State $x_t = \sum_{\tau = 1}^L \bD_{\tau}^{\T} \by_{t-\tau} + \epsilon_t$
\EndFor
\State \jovo{seems like $C$ should be indexed by $t$?}
\end{algorithmic}
\end{algorithm}

