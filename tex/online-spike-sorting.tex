\documentclass{article} % For LaTeX2e
\usepackage{nips13submit_e,times}
\usepackage{hyperref}
\usepackage{url}
\usepackage{natbib}
\usepackage{comment}
\usepackage{amssymb, amsmath}
\usepackage{graphicx}
\usepackage{caption}
\usepackage{subcaption}
\usepackage{algorithm}
\usepackage{algpseudocode}
\usepackage{sidecap}


%\documentstyle[nips13submit_09,times,art10]{article} % For LaTeX 2.09


\title{Real-Time Inference for a Gamma Process \\ Model of Neural Spiking}


\author{
David S.~Hippocampus\thanks{ Use footnote for providing further information
about author (webpage, alternative address)---\emph{not} for acknowledging
funding agencies.} \\
Department of Computer Science\\
Cranberry-Lemon University\\
Pittsburgh, PA 15213 \\
\texttt{hippo@cs.cranberry-lemon.edu} \\
\And
Coauthor \\
Affiliation \\
Address \\
\texttt{email} \\
\AND
Coauthor \\
Affiliation \\
Address \\
\texttt{email} \\
\And
Coauthor \\
Affiliation \\
Address \\
\texttt{email} \\
\And
Coauthor \\
Affiliation \\
Address \\
\texttt{email} \\
(if needed)\\
}

% The \author macro works with any number of authors. There are two commands
% used to separate the names and addresses of multiple authors: \And and \AND.
%
% Using \And between authors leaves it to \LaTeX{} to determine where to break
% the lines. Using \AND forces a linebreak at that point. So, if \LaTeX{}
% puts 3 of 4 authors names on the first line, and the last on the second
% line, try using \AND instead of \And before the third author name.

\newcommand{\fix}{\marginpar{FIX}}
\newcommand{\new}{\marginpar{NEW}}
\newcommand{\bX}{\mathbf{X}}
\newcommand{\bx}{\mathbf{x}}
\newcommand{\dd}{\mathrm{d}}
\newcommand{\Levy}{L\'{e}vy }

%% jovo added stuff
\newcommand{\iid}{\overset{iid}{\sim}}
\newcommand{\mbX}{\mathbf{X}}
\newcommand{\mbY}{\mathbf{Y}}
\newcommand{\Real}{\mathbb{R}}
\providecommand{\mh}[1]{\widehat{#1}}
\providecommand{\mb}[1]{\boldsymbol{#1}}
\providecommand{\mc}[1]{\mathcal{#1}}
\providecommand{\mt}[1]{\widetilde{#1}}
\newcommand{\from}{{\ensuremath{\colon}}}  % :
\usepackage{amsmath,amssymb,amsfonts}
\newcommand{\conv}{\rightarrow}

\newcommand{\efoo}{\end{footnotesize}}
\newcommand{\bfoo}{\begin{footnotesize}}
\renewcommand{\labelenumi}{\theenumi}
\floatname{algorithm}{Procedure}
\renewcommand{\algorithmicrequire}{\textbf{Input:}}
\renewcommand{\algorithmicensure}{\textbf{Output:}}
\floatname{algorithm}{Pseudocode}
\providecommand{\norm}[1]{\left \lVert#1 \right  \rVert}
\newcommand{\T}{^{\ensuremath{\mathsf{T}}}}

\providecommand{\mv}[1]{\vec{#1}}
\providecommand{\mh}[1]{\hat{#1}}
\providecommand{\wh}[1]{\widehat{#1}}
\providecommand{\mhv}[1]{\mh{\mv{#1}}}
\providecommand{\mvh}[1]{\mv{\mh{#1}}}
\providecommand{\mt}[1]{\widetilde{#1}}
\providecommand{\mhc}[1]{\hat{\mathcal{#1}}}
\providecommand{\mbc}[1]{\mb{\mathcal{#1}}}
\providecommand{\mvc}[1]{\mv{\mathcal{#1}}}
\providecommand{\mtc}[1]{\widetilde{\mathcal{#1}}}
\providecommand{\mth}[1]{\mt{\mh{#1}}}
\providecommand{\mht}[1]{\mh{\mt{#1}}}
\providecommand{\mhb}[1]{\hat{\boldsymbol{#1}}}
\providecommand{\whb}[1]{\widehat{\boldsymbol{#1}}}
\providecommand{\mvb}[1]{\vec{\boldsymbol{#1}}}
\providecommand{\sf}[1]{\mathsf{#1}}

\newcommand{\ZZ}{\mathbb{Z}}         
\newcommand{\by}{\mathbf{y}}
\newcommand{\tby}{\mathbf{y}^*}
\newcommand{\ty}{y^*}
\newcommand{\bA}{\mathbf{A}}
\newcommand{\bB}{\mathbf{B}}
\newcommand{\bD}{\mathbf{D}}
\newcommand{\bI}{\mathbf{I}}
\newcommand{\bW}{\mathbf{W}}
\newcommand{\bd}{\mathbf{d}}
\newcommand{\bth}{\mb{\theta}}
\newcommand{\phiv}{\mb{\phi}}
\newcommand{\muv}{\mb{\mu}}

\usepackage{color}
\newcommand{\jovo}[1]{{\color{blue}{\it #1}}}
\newcommand{\dec}[1]{{\color{red}{\it #1}}}
\newcommand{\vr}[1]{{\color{yellow}{\it #1}}}
\newcommand{\eps}{\varepsilon}
\providecommand{\sct}[1]{{\sc \texttt{#1}}}
\newcommand{\smug}{\sct{Opass}}
% \newcommand{\iid}{\overset{iid}{\sim}}

%\nipsfinalcopy % Uncomment for camera-ready version

\setlength{\parsep}{0pt}
\setlength{\headsep}{0pt}
\setlength{\topskip}{0pt}
\setlength{\topmargin}{0pt}
\setlength{\topsep}{0pt}
\setlength{\partopsep}{0pt}

\setlength{\parskip}{2pt}
\parindent 10pt
 \usepackage[compact]{titlesec}
\titlespacing{\section}{20pt}{*0}{*0}
\titlespacing{\subsection}{5pt}{*0}{*0}


\begin{document} 


\maketitle

\begin{abstract}
\smug: Online Real-time Gamma-process Autoregressive Spike-sorting Model

\end{abstract}

\section{Introduction}

The recently announced BRAIN initiative calls for the development of technologies that will advance our understanding of neuroscience \cite{??}.  Crucial to the success of this endeavor will be the advancement of our ability to understand the \emph{dynamics} of the brain, via the measurement of large populations of neural activity at the single neuron level.  Such reverse engineering efforts will benefit from real-time decoding of neural activity, to facilitate adapting the stimuli to probe the functional connectivity more effectively.  Regardless of the experimental apparati used to measure such data (e.g., electrodes or calcium imaging), real-time decoding of individual neuron responses requires real-time spike sorting from large populations of neurons.

Automatic spike sorting methods are continually evolving to deal with more sophisticated experiments.  Most recently, several methods have been proposed to (i) learn the number of separable single units on each electrode or ``multi-trode'' \cite{??}, or (ii) operate online to resolve overlapping spikes from multiple neurons \cite{??}.   To our knowledge, no method to date is able to simultaneously address both of these challenges.  

We develop a fully Bayesian continuous-time generative model of population activity.  More specifically, we posit the existence of a latent marked Poisson process for each neuron.  Previous efforts to address overlapping spiking often assume a fixed kernel for each waveform, but joint intracellular and extracellular recording indicate that this assumption is clearly violated (see Figure \ref{fig:draft}). Thus, we assume that the statistics of the ``mark'' of each Poisson process are time-varying.  Moreover, rather than assuming a priori a fixed number of separable neurons per channel, we place a nonparametric Bayesian prior on the number of neurons.  Collectively we therefore have a completely random measure jointly characterizing spike times and waveforms from a potentially infinite number of neurons \cite{??}.  

Given the above infinite dimensional continuous time stochastic process, we show that it is the limiting process of a discrete time model.  We then develop an online variational Bayesian inference algorithm for this model \cite{??}.  Via numerical simulations, we demonstrate that our inference procedure improves over the previous state-of-the-art, even though we allow the other methods to use the entire dataset for training, whereas we learn online.  Moreover, we demonstrate that we can effective track the time-varying changes in waveform, and detect overlapping spikes.  Indeed, it seems that the false positives detections from our approach have indistinguishable first order statistics from the true positives, suggesting that that second-order methods may be required to reduce false positive rate (i.e., template methods may be inadequate).  Our work therefore suggests that further improvements in real-time decoding of activity may be most effective if directed at simultaneous real-time spike sorting and decoding.  To facilitate such developments and support reproducible research, all code and data associated with this work is provided in the Supplementary materials.


 
% \section{Methods}
% \vspace{-.1in}
\section{Model}
% \vspace{-.1in}

%\nipsfinalcopy % Uncomment for camera-ready version

\subsection{Notation}

Unless otherwise specified, we let lower-case English alphabet characters indicate scalars $x \in \Real$, bold indicates column vectors $\mb{x} \in \Real^p$, and upper-case bold indicates matrices, $\bX \in \Real^{p \times q}$.  Parameters and constants will be Greek characters.  Time will be $t \in (0,T)$, $i \in [N]$ will index the $N$ neurons, where $[N]=\{1,2,\ldots,N\}$. Script will denote sets and pipes will denote the cardinality of the set, e.g. $|\mc{T}|$.  

\subsection{Input}
% \subsection{Data Model}
Our data is a time-series of multielectrode recordings $\bX \equiv (\bx_1, \cdots, \bx_T)$, and consists of $T$ recordings from $C$ channels. 
The set of recording times lie on regular grid with interval length $\Delta$, while $\bx_t \in \mathbb{R}^C$ for all $t$. 
This time-series of electrical activity is driven by an unknown number of neurons {\color{red} and we want to... recap scientific goals}. 
We let the number of neurons be unbounded, though only a few of the infinite
neurons dominate. These neurons contribute the majority of the activity in any finite interval of time; however, as time passes, the total number of 
observed neurons increases {\color{red}(Justify?)}. 
%Each neuron, has its own `shape' A natural model in such a situation is to
The neurons themselves generate continuous-time voltage traces, with the outputs of all neurons superimposed and discretely sampled to produce the 
recordings $\bX$.  At a high level, we model the output of each neuron as a
series of idealized spikes which are smoothed with appropriate kernels, the latter determining the shape of each spike. 
%Each neuron has its own distribution over waveform shapes. 
We describe this in detail, starting first with a model for a single channel recording $\bx\T \equiv (x_1, \cdots, x_T)$.

\subsection{Modeling a single electrode recording}

There is a rich literature characterizing the spiking activity of a single neuron \citep{}, accounting in detail for factors like non-stationarity, 
refractoriness and spike waveform. We shall however make a number of simplifying assumptions (some of which we later relax, others we expect to relax in future work). Figure \ref{fig:schematic} provides a schematic depiction of our generative process.
First, we model the spiking activity of each neuron is 
stationary and memoryless, so that its set of spike times are 
distributed as a homogeneous Poisson process. 
{\color{red} justify?} We model the neurons themselves are heterogeneous, with the $i^{th}$ neuron having
an (unknown) firing rate $\lambda_i$. Call the ordered set of spike times of the $i^{th}$ neuron $\mc{T}_i$; then the time between successive elements of $\mc{T}_i$ is 
exponentially distributed with mean $1/\lambda_i$. We write this as
\begin{align}
  \mc{T}_i &\sim \text{PoissProc}(\lambda_i)
\end{align}
The actual electrical output of a neuron is not binary; instead each spiking event is a smooth perturbation in voltage about a
resting state. This perturbation forms the shape of the spike, and without any loss of generality, we set the resting state to zero. 
{(\color{red} figure? better biological description? comment on how we preprocess the data to get zero mean?)}. 
While the spike shapes vary across neurons as well as across different spikes of the same neuron, each 
neuron has its own characteristic distribution over shapes. 
We let $\bth^*_i \in \Theta$ parametrize this distribution for neuron $i$.
 % having parameter $\bth^*_i$. 
Whenever this neuron emits a 
spike, a new shape is drawn independently from the corresponding distribution. %$p_{\bth_i}$, and 
This waveform is then offset to the time of the spike, and contributes to the voltage trace associate with that spike. The complete detected output of the neuron is the 
superposition of all these spike waveforms plus noise.  
We assume the noise is independent and  identically distributed Gaussian noise at each time step.

% Comment on how this dictionary is obtained now, or in section on inference?)}. 
We model the random spike shapes themselves as weighted superpositions of a dictionary of $K$ basis functions $\bd(t) \equiv (d_1(t), \cdots, d_K(t))\T$. The
dictionary elements are shared across all neurons, and each is a real-valued function of time, e.g., $d_k \in L_2$.
For the $i^{th}$ neuron, the $j^{th}$ spike $\tau_{ij} \in \mc{T}_i$, is associated with a random $K$-dimensional weight vector $\tby_{ij} \equiv (\ty_{ij1}, \ldots \ty_{ijK})\T$, and the 
shape of this spike is given by the weighted sum $\sum_{k=1}^K \ty_{ijk} d_k(t)$. We assume $\tby_{ij} \sim \mc{N}_K(\mb{\mu}^*_i, \Sigma^*_i)$ indicating a $K$-dimensional Gaussian distribution with mean $\mb{\mu}^*_i$ and covariance $\Sigma^*_i$.   Then, at any time $t$, the output of neuron $i$ is
\begin{align}
  x_{i}(t) &= \sum_{j=1}^{|\mc{T}_i|} \sum_{k=1}^K \ty_{ijk} d_k(t - \tau_{ij})
\end{align}



% \begin{center}
% \begin{figure}
% % \includegraphics[width=\textwidth]{../figs/truefalsepositive}
% \caption{Schematic of our Generative Model.}
% \label{fig:schmetic}
% \end{figure}
% \end{center}

{The total signal recorded from any electrode  is the superposition of the outputs of all neurons. Assume for the moment there are $N$
neurons, and define $\mc{T} = \cup_{i \in [N]} \mc{T}_i$ as
the (ordered) union of the spike times of all neurons. 
Let $\tau_l$ indicate the $l^{th}$ overall spike, that is $\tau_l \in \mc{T}$, whereas $\tau_{ij} \in \mc{T}_i$ is the $j^{th}$ spike of neuron $i$.
To map elements  $\tau_{ij} \in \mc{T}_i$ to elements  $\tau_l \in \mc{T}$,   we implicitly have defined a mapping from $(N \times |\mc{T}_i|)$ to $|\mc{T}|$. %, which maps from the $j^{th}$ spikes of neuron $i$ to the $l^{th}$ overall spike.

Let $\nu_l \in N$ be the neuron to which the $l^{th}$ element of $\mc{T}$ belongs, 
and let $\bth_l \equiv (\mb{\mu}_l, \Sigma_l)$ be the neuron parameter associated with
that spike. 
Thus $\bth_l = \bth^*_{\nu_l}$. 
% Furthermore, let $p_j \from \mc{T} \to \mc{T}_j$ index the position of spike $j$ of $\mc{T}$ in $\mc{T}_{\nu_j}$.
% , the spike train of neuron $\nu_j$. 
Thus $\tau_l = \tau_{\nu_l j}$ is the $j^{th}$ spike of neuron $\nu_l$. Finally, define $\by_l=(y_{l1}, \ldots, y_{lK})\T \equiv \tby_{\nu_j j}$ as the weight vector of spike $\tau_l$. Then, we have that}
\begin{align}
  x(t) &= \sum_{i \in [N]} x_{i}(t) =   \sum_{l \in |\mc{T}|} \sum_{k \in [K]} y_{lk} d_k(t - \tau_{l}), \label{eq:spk_sup}
\intertext{where}
  \by_{l} & \sim \mc{N}_K(\mb{\mu}_{l}, \Sigma_{l}). \label{eq:spk_shape}
\end{align}
% 
From the superposition property of the Poisson process \citep{kingman93}, the overall spiking activity $\mc{T}$ is a 
Poisson process with rate $\Lambda = \sum_{i \in [N]} \lambda_i$. Each event $\tau_l \in \mc{T}$ is associated with a pair of labels, the parameter of the neuron to which it 
is assigned ($\bth_l = (\mb{\mu}_l, \Sigma_l)$), and the weight-vector characterizing the spike shape ($\by_l$). We can view these as the ``marks'' of a 
marked Poisson process $\mc{T}$.  From the properties of the Poisson process, we have that the marks $\bth_l$ are drawn i.i.d. from a probability measure 
\begin{align}
 G(\dd \bth) = \frac{1}{\Lambda}\sum_{i \in [N]} \lambda_i \delta_{\bth^*_i}    \label{eq:mark_distr}
\end{align}
Note that with probability one, the neurons have distinct parameters, so that the mark $\bth_l$ associated with spike $l$ identifies the
neuron which produced it: $G(\bth_l = \bth^*_i) = P(\nu_l= i) = \lambda_i/\Lambda$. Given $\bth_l$, $\by_l$ is distributed as in
Eq.~\ref{eq:spk_shape}. The output waveform $x(t)$ is then a linear functional of this marked Poisson process (Eq.~\eqref{eq:spk_sup}). 

\subsection{Completely random measures (CRMs)}

The previous section assumed a known number of neurons $N$. In practice however, our recordings are a superposition of the outputs of an unknown
number of neurons. We deal with this uncertainty by taking a nonparametric Bayesian approach, and letting the number of neurons $N$ tend to infinity
{\color{red} Can we provide a biogical/electrical justification of this?}. 
Such an approach leads to an elegant and flexible modeling framework, and has  already proved successful in neuroscience applications
\citep{WoodBla2008}.
Since only a finite number of spikes are observed in any finite interval, the total rate $\Lambda$ must 
also be finite; moreover, as we described earlier, we want this to be dominated by a few $\lambda_i$. 
A natural framework that captures these  modeling requirements is that of completely random measures \citep{Kingman:PJM67}.
Completely random measures are stochastic processes that form flexible and convenient priors over
infinite dimensional objects like probability distributions \citep{JamesLP09}, hazard functions \citep{Hjo1990}, latent features \citep{ThiJor2007} etc. 
These have been well studied in the Bayesian nonparametrics and machine learning communities, and there exists a wealth of literature on
their theoretical properties, as well as on computational approaches to posterior inference.

Recall that each neuron is characterized by a pair of parameters $(\lambda_i, \bth^*_i)$; the former characterizes the distribution over spike times, 
and the latter over spike
shapes. With Eq.~\eqref{eq:mark_distr} in mind, we map the infinite collection of pairs $\{(\lambda_i, \bth^*_i)\}$ to an atomic measure on $\Theta$:
\begin{align}
  \Lambda(\dd \bth) = \sum_{i=1}^{\infty} \lambda_i \delta_{\bth^*_i}
\end{align}
For any subset $\varTheta$ of $\Theta$, the measure $\Lambda(\varTheta)$ equals \( \sum_{\{ i: \bth^*_i \in \varTheta \} } \lambda_i\). We allow $\Lambda(\cdot)$ to be random,
modeling it as a realization of a completely random measure. Such a random measure has the property that for any two disjoint subsets $\varTheta_1$,  $\varTheta_2 \subseteq \Theta$, the measures $\Lambda(\varTheta_1)$ and $\Lambda(\varTheta_2)$ are independent. 
This distribution over measures is induced by a distribution
over the infinite sequence of weights (the $\lambda_i$'s), and a distribution over the sequence of their locations (the $\bth^*_i$'s). 
For a CRM, the weights $\lambda_i$ are the jumps of a \Levy process \citep{Sato90}, and their distribution is characterized by a 
\Levy measure $\rho(r)$. The locations $\bth^*_i$ are drawn i.i.d.\  from a base probability measure $H(\bth^*)$.
As is typical, we assume these to be independent (though this is not necessary). {\color{red} if there's space, I
can elaborate on the construction of the CRM from its Levy measure, though this is not necessary}

The particular class of CRM is determined by the \Levy measure $\rho(r)$. For our application, we set $\rho(r) = r^{-1}\exp(-r\alpha)$;
this results in a CRM called the Gamma process ($\Gamma$P) \citep{applebaum2004}. 
The Gamma process has the convenient property that the 
total rate $\Lambda \equiv \Lambda(\Theta) = \sum_{i=1}^{\infty} \lambda_i$ is Gamma distributed (and thus conjugate to the Poisson process prior on $\mc{T}$).
%\footnote{We abuse notation by using $\Lambda$ to denote both the measure as well as the total measure of $\Theta}
%The Gamma distribution has shape parameter $1$ and scale parameter $\alpha$.  Since this is finite almost surely, so too is $\mc{T}$. 
The Gamma process is also closely connected with the Dirichlet process \citep{Ferguson73}, which will prove useful
later on.
Other choices of the \Levy intensity can be used to capture greater uncertainty in the number of neurons active in any finite interval, or to model
power-law behavior in the number of spikes emitted by different neurons.

To complete the specification on the Gamma process, we need to choose a base-measure $H(\bth^*)$.
Recalling that $\bth^* \equiv (\mu^*, \Sigma^*)$ gives the mean and variance of the weight-vector $\by$ of each neuron, we set $H(\bth^*)$ 
to be the conjugate normal-Wishart distribution. Our overall model is then:
\begin{subequations}
\begin{align}
  \Lambda(\cdot) & \sim \Gamma \text{P}(\alpha, H(\cdot)) \\ %\mathcal{NW}(\mu, \Sigma)) \\
  \mc{T}_i\ \  &\sim \text{PoissProc}(\lambda_i) \quad i \in \ZZ \\
  \by_{ij} & \sim \mc{N}_K(\mb{\mu}^*_i, \Sigma^*_i) \quad i,j \in \ZZ \\
  x_i(t) &= \sum_{j = 1}^{|\mc{T}_i|}  \sum_{k = 1}^{K} \tilde{y}_{ijk} d_k(t - \tau_{ij}) \\
  x(t)   &= \sum_{i=1}^{\infty} x_i(t)
\end{align}
\end{subequations}
%Each spike of each neuron is associated with a time $e$ and a weight vector $y$, and one can view the model above as a doubly stochastic Poisson
%process on the product space. 
% 
It will be more convenient to work with the marked Poisson process representation of Eqs.~\ref{eq:spk_sup} and \ref{eq:spk_shape}. 
The superposition process $\mc{T}$ is a rate $\Lambda$ Poisson process,
and under a Gamma process prior, $\Lambda$ has a Gamma distribution with shape and scale parameters $1$ and $\alpha$ respectively \citep{Ferguson73}.
As we saw (Eq.~\eqref{eq:mark_distr}), the labels $\bth_i$ assigning events to neurons are drawn i.i.d. from a normalized Gamma 
process $G(\dd \bth)$:
\begin{align}
 G(\dd \bth) = \frac{1}{\Lambda} \sum_{l=1}^{\infty} \lambda_l
\end{align}
$G(\dd \bth)$ is a random probability measure called a \emph{normalized random measure} \citep{JamesLP09}. Importantly, a 
normalized Gamma process is the Dirichlet process (DP) \citep{Ferguson73}, so that $G$ is a draw from a Dirichlet process. For the $l^{th}$ spike in $\mc{T}$, given its 
parameter $\bth_l$, its shape vector is drawn from a normal distribution
with parameters $(\mb{\mu}_{l}, \Sigma_{l})$. Thus the spike parameters $\bth$ are i.i.d.\ draws from a Dirichlet process, while the weight vectors are
draws from a DP mixture model of Gaussians (DPMM) \citep{Lo1984}.

The connection with the DP allows us to simplify the representation of our model. In particular, we exploit a remarkable property of the DP that
allows us to integrate out the infinite-dimensional variable $G(\cdot)$. The resulting marginal distribution over observations follows the so-called
 Chinese restaurant process (CRP) \citep{Pit2002a}. Under this scheme, the $l^{th}$ spike is assigned the same parameter as an earlier spike with probability 
proportional to the number of earlier spikes having that parameter. It is assigned a new parameter (and thus, a new neuron is observed) with probability 
proportional to $\alpha$. Letting $C_t$ be the number of neurons observed until time $t$, and  $|\mc{T}_i|$ be the number of spikes produced by neuron $i$,
we then have for spike $l$ at time $t = \tau_l$: 
\begin{align}
  P({\nu_l} = i) & \propto 
  \begin{cases}
   |\mc{T}_i| \quad i \in \{1,\cdots, C_{t}\} \\
   \alpha \quad\ i = C_{t} + 1 \\
  \end{cases}  
\label{eq:crp_marg_pr}
\end{align}
%\footnote{Strictly speaking, the process we just described is called a P\'olya urn scheme \citep{BlaMac1973}; for simplicity, we do not distinguish between this and the CRP.}.
and $\bth_l = \bth^*_{\nu_l}$. 
The marginalization property of the DP allows us to integrate out the infinite-dimensional rate vector $\Lambda(\cdot)$, and sequentially 
assign spikes to neurons based on the assignments of earlier spikes.
%is assigned (or equivalently, the parameter $\bth$ associated with that neuron). These marks are drawn from a probability measure 
%$G(\dd \bth) = \frac{1}{R} R(\dd \bth)$. From the properties of the Gamma process, the probability measure $G$ a Dirichlet process, 
To do so, we need one final property: for the Gamma process, the random probability measure $G(\cdot)$ is independent of the total mass $\Lambda(\Theta)$. 
Consequently, the clustering of spikes (determined by $G(\cdot)$) is independent of the rate $\Lambda$ at which they are produced. We then have
 the following model equivalent to the one above:
\begin{subequations}
\begin{align}
  \Lambda & \sim \text{Gamma}(1,\alpha) \\
  \mc{T} &\sim \text{PoissProc}(\Lambda) \\
  \bth_l &\sim \text{CRP}(\alpha, H(\cdot)), \quad l \in [|\mc{T}|]   \label{eq:CRP}\\
  \by_l &\sim \mc{N}_K(\mb{\mu}_{l}, \Sigma_{l}), \quad  l \in [|\mc{T}|]   \label{eq:CRP_mix}\\
  x(t) &=   \sum_{l=1}^{|\mc{T}|} \sum_{k=1}^K y_{lk} d_k(t - \tau_{l})
\end{align}
\end{subequations}
Unlike most applications which observe the outputs of a CRP, \jovo{something weird here} out observation at any time $t$ a convolution-like function of the CRP outputs of all
earlier times. Consequently, we cannot directly apply standard techniques for posterior inference. In \S \ref{sec:inf}, we develop a novel online 
algorithm for posterior inference; first, we provide a discrete-time approximation to our model.

%For neuron $i$, the sequence of spike times is distributed as a Poisson process with random rate $\lambda_i$.
%Each event $\tau_{ij}$ is associated with a mark or label $y_{ij}$ drawn from a normal distribution (again, with random parameters).
%More broadly, we can view the superposed process $\mc{T}$ as a rate $\Lambda$ Poisson process, with each event having a pair of marks, the neuron identity $i$,
%and weight $y$. From the properties of the Gamma process, the pair form a draw from a Dirichlet process.
%Our data is in a form that makes discrete-time modeling more natural, and an approach now is
%one based on the Beta process-binomial process.

\subsection{A discrete-time approximation}
In the previous subsections, we modeled the continuous-time voltage waveform output of a neuron. Our data on the other hand consists of recordings
at a discrete set of times. While it is possible to make inferences about the continuous-time process that underlies these discrete recordings,
in this paper, for simplicity, we restrict ourselves to the discrete case. We thus provide a discrete-time approximation to the model above. 
Our approximation follows easily from the marked Poisson process characterization of the model.

Recall first the Bernoulli approximation to the Poisson process: a sample from a Poisson process with rate $\Lambda$ can be approximated by discretizing
time at a granularity $\Delta$, and assigning each bin an event independently with probability $\Lambda\Delta$. The accuracy of this approximation increases 
as $\Delta$ tends to $0$.
%
%This suggests the following approximation at a time resolution $\Delta$. Draw the random Poisson process rate $\Lambda$ drawn from a Gamma$(1,\alpha)$ 
%distribution. Simultaneously, draw a random probability measure
% $G$ from a Dirichlet process. Assign an event to an interval independently with probability $\Lambda\Delta$, and to each event, assign a random mark drawn 
In our case, we have to approximate the marked Poisson process $\mc{T}$. All this requires additionally is to assign marks $\bth_i$ and $\by_i$ to each event 
in the Bernoulli approximation. Following Eqs.~\eqref{eq:CRP} and \eqref{eq:CRP_mix}, the $\bth_l$'s are distributed according
to a Chinese restaurant process, while each $\by_l$ is drawn from a normal distribution parametrized by the corresponding $\bth_l$. We discretize the 
elements of dictionary $d_k \equiv \{d_k(t)\}_{t \in (0,T)}$ as well, defining a mapping from $L_2$ to $\Real^L$ yielding discrete dictionary elements $\mt{d}_k=(\mt{d}_k[1], \ldots, \mt{d}_k[L])\T$,  so  $\bD \in \Real^{K \times T}$. The shape of the $j^{th}$ spike is now a vector of length $L$, and for a weight vector
$\by$, is given by $\bD \by$.

\subsection{Correlations in time and across electrodes}
We now describe three extensions to the model outlined previously. The first is the inclusion of measurement noise: 
{\color{red} Biology?}. Let $\eps_t$ be the noise at time $t$, we model this as independent, additive and Gaussian.
%However, rather than modeling the noise as independent across time, we model it as a first-order autoregressive process. This can capture
%effects like the movement of electrodes during the experiment. 

Our second extension allows the parameters of each neuron to evolve with time {\color{red} Justify, eg cells dying}. Recall that for neuron $i$, the 
associated parameters are ($\mu^*_i, \Sigma^*_i$). We keep the covariance $\Sigma^*_i$ fixed, however we model the mean $\mu^*_i$ as a first-order
autoregressive process. 

Finally, all we have described so far is the input to a single electrode, our observations on the other hand are multielectrode recordings. 
A simple approach has all electrodes having the same signal (but independent noise); we relax this by only requiring the signals to be
correlated across electrodes {\color{red} Need to check how we set details of this correlation matrix}.


% \vspace{-.1in}
\section{Inference} \label{sec:inf}
% \vspace{-.1in}
\newcommand{\tx}{\tilde{x}}

%       \begin{algorithm}
%       \caption{State-dependent thinning for sMJPs}\label{alg:smjp_unif}
%       \begin{tabular}{p{1.2cm}p{12.4cm}}
%       Input:&  Hazard functions $A_{ss'}(\cdot)\ \forall s,s' \in \mathcal{S}$, and an initial distribution over states $\pi_0$. \\ 
%        & Dominating hazard functions $B_{s}(\tau) \ge A_{s}(\tau)\ \forall \tau, s$, where $A_{s}(\tau) = \sum_{s'} A_{ss'} (\tau)$. \\
%       Output:& A piecewise constant path $(V,L,W) \equiv (v_i, l_i, w_i)$ on the interval $[t_{start},t_{end}]$. % defined by a set of state and time pairs.
%       \end{tabular}
%       \begin{algorithmic}[1]
%       \State Draw $v_0 \sim \pi_0$ and set $w_0 = t_{start}$. Set $l_0 = 0$ and $i = 0$.
%       %\State Set $A_{s_i}(\tau) = \sum_j A_{s_i,j} (\tau)$ and define $u_{s_i}(\tau) \ge A_{s_i}(\tau) \forall \tau$. \label{alg:loop}
%       %\State Let $\tau_o = (w_{i} - l_i)$. \label{alg:smjp_loop}
%       \While{$w_i < t_{end}$}
%       \State Sample $\tau_{hold} \sim B_{v_i}(\cdot), \text{\ with } \tau_{hold} > l_i$.
%              Let $\Delta w_i = \tau_{hold} - l_i$, and $w_{i+1} = w_i + \Delta w_i$. 
%       %\State Sample $\tau_n \sim B_{v_i}(\tau_n) \exp \left( -\int_{\tau_o}^{\tau_n} B_{v_i}(\tau) d\tau \right), \text{\ conditioned on } \tau_n > \tau_o$.
%       %       Let $w_{i+1} = \tau_n + l_i$. 
%       \If{$\frac{A_{v_i}(\tau_{hold})}{B_{v_i}(\tau_{hold})}$}   \label{enum:thin}
%         \State Set $l_{i+1} = 0$, and sample $v_{i+1}$, with $P(v_{i+1} = s'|v_i) \propto A_{v_is'}(\tau_{hold}),\ s' \in \mathcal{S}$.   \label{enum:sample}
%       \Else
%         \State Set $l_{i+1} = l_i + \Delta w_i$, and $v_{i+1} = v_i$.
%       \EndIf
%       %\State Define $q_j = \frac{A_{v_ss'}(\tau_n)}{B_{v_i}(\tau_n)}\ \forall j \in \mathcal{S}$ and $q_0 = 1 - \sum_{j \in  \cS} q_j$.% = 1 - \frac{A_{v_i}(\tau)}{u_{v_i}(\tau)}$.
%       %\State Sample $v_{i+1} \sim q$. If $v_{i+1} \neq 0$, set $l_{i+1} = w_{i+1}$; else set $v_{i+1}$ to $v_i$ and $l_{i+1} = l_i$.
%       \State Increment $i$.
%       \EndWhile
%       \State Set $w_{|W|} = t_{end}$, $v_{|W|} = v_{|W|-1}$, $l_{|W|} = l_{|W|} + w_{|W|} - w_{|W|-1}$. 
%       \end{algorithmic}
%       \end{algorithm}


We now address the problem of posterior inference over the latent variables gives the matrix $\bX$ of multielectrode recordings. Unsurprisingly, exact 
inference is intractable, and we have to resort to approximating the posterior distribution.
There exists a vast literature on approximate inference for Bayesian nonparametric models, especially so for models based on the Dirichlet process.
Traditional approaches are sampling-based, typically involving Markov chain Monte Carlo techniques (see eg.\ \citep{Nea2000, IshJam2001}), 
and recently there has also been work on constructing deterministic approximations to the intractable posterior (eg.\ \citep{BleJor2006, MinGha2003}).
Our problem is complicated by two factors. The first is the convolutional nature of our observation process, where we observe a functional of the 
sequence of observations drawn from the DP mixture model. This is in contrast to the usual situation where one directly observes the DPMM outputs themselves.
The second complication is a computational requirement: typical inference schemes are batch methods that are slow and computationally expensive. 
Our ultimate goal, on the other hand, is to perform inference in real time, so that these batch approaches are unsuitable for our purposes.

With the latter objective in mind, we develop an online algorithm for posterior inference. Our algorithm is based on \cite{WangDun2009}, though this was 
developed for the usual case of i.i.d.\ observations drawn from a DPMM. We generalize this algorithm to our observation process, and also 
to account for the time-evolution of the cluster parameters and the Markov nature of the observation noise.

At a high level, at time $t$, our algorithm maintains the set of times of the spikes it has inferred from the observations until time $t$. It also maintains
the identities of the neurons that it assigned each of these spikes to, as well as the weight vectors determining the shapes of the associated spike 
waveforms. In addition to these point estimates, the algorithm also maintains a set of posterior distributions $q_{it}(\theta^*_i)$ where $i$ spans over the
set of neurons associated with the spikes seen so far. Let the number of unique neurons observed at time $t$ be $C_t$, so that $i$ varies from $1$ to $C_t$.
For each $i$, $q_{it}(\theta^*_i)$ approximates the distribution over the parameters 
$\theta_i^* \equiv (\mu_i^*, \Sigma_i^*)$ of neuron $i$ given the observations until time $t$. 
Having identified the location and shape of spikes from earlier times, we can calculate their contribution to the recording $x_{t+1}$ at time $t+1$.
We subtract this term from $x_{t+1}$, and treat the residual $\tx_{t+1}$ as an observation from a DP mixture model.
Given this residual, our algorithm then makes a hard decision about whether or not this was produced by an underlying spike, what neuron that spike belongs 
to (one of the earlier neurons or a new neuron), and what the shape of the associated spike waveform is. The latter is used to calculate
$q_{i,t+1}(\theta^*_i)$, the new distribution over neuron parameters at time $t+1$. Our algorithm proceeds recursively in this manner. 
We describe it in more detail below.


Recall that each spike waveform spans $L$ time bins. 
Let $z_t$ indicate whether or not a spike is present at time $t$, with $\by_t$ giving its shape, and $\nu_t$ is associated neuron. 
The residual at time $t$ is then given by
\begin{align}
  \tx_t = x_t - \sum_{i=1}^L \bA\by_{t-i} \label{eq:resid}
\end{align}
Given this residual, the first step is to decide whether or not there is a spike underlying this residual.
By Bayes' rule,
\begin{align}
  P(z_t = 1 | \tx_t)  &\propto P(z_t = 1, \tx_t) = \sum_{\nu_t = 1}^{C_{t-1}+1} P(\tx_t, {\nu_t} | z_t = 1) P(z_t = 1) \\
\intertext{Here, $C_{t-1}$ is the number of unique neurons underlying the observations before time $t$. Letting $n_i$ be the number of spikes from neuron 
$i$, we have from the Chinese restaurant process that}
  P({\nu_t} = i | z_t = 1) & \propto 
  \begin{cases}
   n_i \quad i \in \{1,\cdots, C_{t-1}\} \\
   \alpha \quad\ i = C_{t-1} + 1 \\
  \end{cases}  \label{eq:crp_marg}\\
\intertext{On the other hand,}
  P(\tx_t | {\nu_t = i} , z_t = 1) &= \int_{\Theta} P(\tx_t | \theta_t) q_{it} (\theta_t) \dd \theta_t  \label{eq:norm_nw}
\end{align}
Recall that  $P(\tx_t | \theta_t)$ is just the normal distribution with mean $\mu_t$ and variance $\theta_t$. We let $q_{it}(\cdot)$ be the normal-Wishart
distribution. % with parameters 
We can then evaluate integral \eqref{eq:norm_nw}, and then summation \eqref{eq:crp_marg} to calculate $P(z_t = 1 | \tx_t)$. 
If this exceeds a threshold of $0.5$ we decide that there is a spike present at time $t$, otherwise, we set $z_t = 0$.
Note that we make this decision after marginalizing over all possible cluster assignments and all values of the weight vector $\by_t$.
In the event that we decide that a spike is present , we simplify matters collapsing these parameters, and the spike the neuron and weight-vector
with highest posterior probability. Given these two, we update the posterior distribution over parameters of cluster $\nu_t$, thereby obtaining
$q_{i,t+1}$.



\section{Experiments}
\subsection{Publicly Available Data}
Experiments were performed on a subset of the publicly available\footnote{http://crcns.org/data-sets/hc/hc-1/} dataset described in \cite{Henze2000}.  We used the dataset d533101 that consisted of an extracellular tetrode and a single intracellular electrode.  The recording was made simultaneously on all electrodes and was set up such cell with the intracellular electrode was also recorded on the extracellular array. 

The intracellular recording is relatively noiseless and gives nearly certain firing times of the intracellular neuron.  The extracellular recording contains the spike waveforms from the intracellular neuron as well as an unknown number of additional neurons.  The data is a 4-minute recording at a 10kHz sampling rate and preprocessed with a high-pass filter at 800 Hz.

Each algorithm gives a clustering of the detected spikes.  In this dataset, we only have a partial ground truth, so we are only able to analyze whether a spike comes from the intracellular (IC) recording or not.  In these experiments, we define a detected spike to be an IC spike if the IC recording has a spike within .5ms of the detected spike.  We define the cluster with the greatest number of intracellular spikes as a the "IC cluster."  Figure \ref{hc1res} shows the performance on the intracellular cluster versus competing methods.  The single-channel experiments were all run on channel 2 (the results were nearly identitcal for all channels).  The spike detections for the offline methods were given by using a threshold at 3 times the noise standard deviation \cite{Lewicki}. (Need to do more comparisons and descriptions) and windowed at a size $P=30$ (check if $P$ matches with Vinayak notation).  The action potential window was set at 30, and PCA was used to reduce the space to $K$=3 for the experiments.  The results were very similar for $K$=2, $K$=3, and $K$=4.   There were 3742 spikes detected with the threshold, and 753 of those corresponded to the intracellular spikes.

The online algorithms were all run with weakly informative parameters (add parameters once I get vinayak's notation). The parameters were insensitive to minor changes.  Running time in unoptimized MATLAB code for 4 minutes of data was 31s was a single channel and 3 minutes for all 4 channels on a 3.2GHzIntel Core i5 machine with 6GB of memory.  We show the inferred $y_k$ fordetected spikes in the 2-PCA space in Figure \ref{pcaonlinear}.

As as been demonstrated previously (cite paninski), the waveform shape of a neuron may change over time.  The mean waveform over time for the intracellular neuron is shown in Figure \ref{evohc1}.  For the IC cluster, it is of interest to investigate the properties of the true positives, the false positives, as well as the IC spikes that are missed by the algorithm.  Errorbar plots for these groups are shown in Figure \ref{truewaveforms}; this figure demonstrates the great similarity of the true positives and the false positives, while the missed positives (spike waveforms not detected or not in the IC cluster) have high variability and a different mean shape.

To get an idea of the robustness of the algorithm, we split the data into random segments of 2minutes, and ran the algorithms on that dataset.  Add plot and information.
\section{Multi-Electrode Sensors}
In the tetrode case the spike undoubtedly appears on all channels at once.  Often, people will just concatenate the channels to process the data (eg. cites).  When the action potential will only appear on a subset of channels it is nice to allow the action potential to vary in a low-dimensional subset in each of the channels instead of a low-dimensional subset over all the channels.  This is especially important when the appropriate dictionary is not known beforehand.

We use processed data from novel NeuroNexus devices.  (Need to give a description of this).  The first device we consider is a set of 3 channels of data shown in Figure \ref{3dev}.  The neighboring electrode sites in these devices have 30 $\mu$m between electrode edges and 60 $\mu$m between electrode centers.  These devices are close enough that a locally-firing neuron could appear on multiple electrode sites (cite that paper on action potential overlap), and neighboring channels warrant joint processing.

 The top 3 clusters found in the first 10 minutes of data are shown in Figure (add in a bit).


\begin{center}
\begin{figure}
\begin{subfigure}[b]{.5\textwidth}
\centering
\includegraphics[width=\textwidth]{../figs/truefalsepositive}
\caption{}
\label{hc1res}
\end{subfigure}
\begin{subfigure}[b]{.5\textwidth}
\includegraphics[width=\textwidth]{../figs/ykarreduced}
\caption{}
\label{pcaonlinear}
\end{subfigure}
\caption{Results on the d533101 dataset.  (a) This shows the number of true positives versus the number of false positives.  Our methods do better, yay! (b) Results on the d533101 dataset from the online algorithm with an AR mean}
\end{figure}
\end{center}
\begin{center}





\begin{figure}
\begin{subfigure}[b]{.5\textwidth}
\includegraphics[width=\textwidth]{../figs/evohc1}
\caption{}
\label{evohc1}
\end{subfigure}
\begin{subfigure}[b]{.5\textwidth}
\includegraphics[width=\textwidth]{../figs/IntracellularTrueFalsePositivesv2}
\caption{}
\label{truewaveforms}
\end{subfigure}



\caption{(a) Mean IC waveforms over time.  Each colored line represents the mean of the waveform averaged over 24s and the color gives the elapsed time.  This neuron decreases in amplitude over the period of the recording. (b) Errorbar plots of the true positives and the false positives in the IC cluster.  While the false positives have slightly more variability, the mean shape for the false positives and the true positives is nearly identical.  The true misses have a significantly lower amplitude as well as high variability}
\end{figure}
\end{center}
\begin{center}
\begin{figure}
\begin{subfigure}[b]{.24\textwidth}
\includegraphics[width=\textwidth]{../figs/3dev}
\caption{}
\label{3dev}
\end{subfigure}
\begin{subfigure}[b]{.24\textwidth}
\includegraphics[width=\textwidth]{../figs/3devim/clus1}
\caption{}
\end{subfigure}
\begin{subfigure}[b]{.24\textwidth}
\includegraphics[width=\textwidth]{../figs/3devim/clus2}
\caption{}
\end{subfigure}
\begin{subfigure}[b]{.24\textwidth}
\includegraphics[width=\textwidth]{../figs/3devim/clus3}
\caption{}
\end{subfigure}
\caption{Device used. Channels in large, red numbers.}
\end{figure}
\end{center}

 
\section{Discussion}
% \paragraph{Summary}

We have developed a novel Bayesian nonparametric model for spike sorting called \smug.  Our model and inference procedure incorporate certain features that previous approaches---be they nonparametric or not---lacked.  Most importantly, we developed a variational Bayesian online inference scheme, that enabled faster than real-time posterior inference.  Such computational efficiency is crucial for sequential experimental design \cite{}.  Although we only provided our algorithm with streaming data, \smug\ outperformed all competitor \emph{batch} algorithms (that is, algorithms that consume all the data at once). Our improved sensitivity and specificity seem to arise from multiple sources including (i) improved detection, (ii) correlated noise, (iii) capturing overlapping spikes, (iv) tracking waveform dynamics, and (v) utilizing multiple channels.  While others have developed closely related Bayesian models for clustering \cite{WoodBla2008,wood2009}, deconvolution based techniques \cite{Pillow2013}, time-varying waveforms \cite{calabrese2011kalman},  \emph{or} online methods \cite{OSORT, Franke2010}, we are the first to our knowledge to incorporate \emph{all} of those features.

An interesting implication of our work is that it seems that our errors result from problems of the dictionary.  Fig.\ \ref{fig:ICold} shows the true positives, missed positives, and false positives in the space of the first two PCs (see Supplementary Fig.\ \ref{pairs}). Although we are clustering in a higher dimensional subspace (5 PCs) 

% \vspace{-10pt}
\paragraph{Limitations of Dictionary-Based Techniques} \label{sub:template}
\vspace{-5pt}


\begin{center}
\begin{figure}[h!]
\begin{subfigure}[b]{.33\textwidth}
\includegraphics[width=\textwidth]{../figs/new/ICclusteroldpca.pdf}
\caption{}
\label{fig:ICold}
\end{subfigure}
\begin{subfigure}[b]{.33\textwidth}
\includegraphics[width=\textwidth]{../figs/new/ICclusternewpca.pdf}
\caption{}
\label{fig:ICnew}
\end{subfigure}
\begin{subfigure}[b]{.33\textwidth}
\includegraphics[width=\textwidth]{../figs/IntracellularTrueFalsePositivesv2}
\caption{}
\label{truewaveforms}
\end{subfigure}
\caption{Template matching.
(c) Errorbar plots of the true positives and the false positives in the IC cluster.  While the false positives have slightly more variability, the mean shape for the false positives and the true positives is nearly identical.  The true misses have a significantly lower amplitude as well as high variability. 
} \label{fig:IC-PCA}
\end{figure}
\end{center}




in discussion:

embarrassingly parallel per 4

ignore time steps that aren't useful

let $\lambda_i$ vary as a function of: (i) spike histories, (ii) stimulus, (iii) possibly baseline drift?


\clearpage
\section{comments}

{\color{red} perhaps add comments about time-evolution and the false positives we avoid by using multi-channel analysis}

\jovo{add grids to all panels of all figs by default, remove if it looks shitty.}

\jovo{@dec - for fig 1 keep colors/symbols the same in the two panel, if possible.  maybe use a scheme where color indicates algorithm and shape indicates \# of components.  also, be consistent about "IC" cluster vs. "Intracellular" Cluster. and normalize, renaming axes XY Rate instead of only XY. (b) title should be ``ROC Curve Comparisons''}  






\begin{comment}
\subsubsection*{Acknowledgments}

Use unnumbered third level headings for the acknowledgments. All
acknowledgments go at the end of the paper. Do not include 
acknowledgments in the anonymized submission, only in the 
final paper. 
\end{comment}

\clearpage
{\small
\bibliography{refs}
\bibliographystyle{unsrt}
}

\clearpage
\appendix

\section{Supplementary Text}

Our overall model is then:
\begin{subequations}
\begin{align}
  \mc{T}_i\ \  &\sim \text{PP}(\lambda_i) \quad i \in \NN, \quad &\text{ where } \Lambda(\cdot)&=\sum_{i=1}^{\infty} \lambda_i \delta_{\theta^*_i} \sim \Gamma \text{P}(\alpha, H(\cdot| {\phi})), \\ %\mathcal{NW}(\mu, \Sigma)) \\ \\
\vspace{-.8in}
  x_i(t) &= \sum_{j = 1}^{|\mc{T}_i|}  \sum_{k = 1}^{K} y^*_{ijk} \mathsf{d}_k(t - \tau_{ij}), \quad &\text{ where }\by^*_{ij}  &\sim \mathsf{N}_K(\mb{\mu}^*_i, \Sigma^*_i) \quad i,j \in \NN, \\
  x(t)   &= \sum_{i=1}^{\infty} x_i(t) + \eps_t, \quad &\text{ where at any time $t$, } \eps_t &\sim \mathsf{N}(0,\Sigma_x) \text{ independently}
\end{align}
\end{subequations}

\begin{algorithm}
\caption{Generative mechanism for the multi-electrode, non-stationary, discrete-time process}\label{alg:gen_proc}
\begin{tabular}{p{1.2cm}p{12.4cm}}
Input:&  a) the number of bins $T$, and the bin-width $\Delta$\\
  &  b) the $K$-by-$L$ dictionary $\bD$ of $K$ basis functions\\
  &  c) the DP hyperparameters $\alpha$ and $\phi$.\\ 
  &  d) the transition matrix $\bB$ of the neuron AR process \\
Output:& \  An $M$-by-$T$ matrix $\bX$ of multielectrode recordings. % defined by a set of state and time pairs.
\end{tabular}
\begin{algorithmic}[1]
\State Initialize the number of clusters $C_1$ to $0$.
\State Draw the overall spiking rate $\Lambda \sim \text{Gamma}(\alpha, 1)$.
%\State Set $A_{s_i}(\tau) = \sum_j A_{s_i,j} (\tau)$ and define $u_{s_i}(\tau) \ge A_{s_i}(\tau) \forall \tau$. \label{alg:loop}
%\State Let $\tau_o = (w_{i} - l_i)$. \label{alg:smjp_loop}
\For{$t$ in $[T] $}
\State Sample $\mt{z}_t \sim \text{Bernoulli}(\Lambda \Delta)$, with $\mt{z}_t = 1$ indicating a spike in bin $t$.
\If{$\mt{z}_t = 1$}   \label{enum:thin}
  \State Sample $\mt{\nu}_t$, assigning the spike to a neuron, with
%\begin{align*}
$  \mathsf{P}({\mt{\nu}_t} = i) \propto 
  \begin{cases}
   |\mc{T}^t_i| \quad i \in [C] \\
   \alpha \quad\ i = C + 1 \\
  \end{cases}$
%\end{align*}
       \If{ $\nu_t = C_t + 1$} 
          \State  $C_{t+1} \leftarrow C_t + 1$. 
		\State Set $\theta^*_{C_{t+1}} \sim H_{\phi}(\cdot)$, and $\mc{T}_{C_{t+1}}$=\{t\}.
       \Else \State  $\mc{T}_{\nu_t} \leftarrow \mc{T}_{\nu_t} \cup \{t\}$.
    \EndIf
\State Set $\theta_t = \theta^*_{\nu_t}$, recalling that $\theta_t \equiv (\mb{\mu}_t, \Sigma_t)$.
\State Sample $\by_t = (\by^1_t; \cdots; \by^M_1) %\equiv (\by_{t1}, \cdots, \by_{t\Upsilon}) 
           \sim \mathsf{N}(\mb{\mu}_t, \Sigma_t)$, determining the spike shape at all electrodes.
%\State $\bx^h_{t:t+L} = A\by^h$
\State $ x^m_t = \sum_{h = 1}^L \bD_{:,h}^{\T} \by^m_{t-h-1} + \epsilon^m_t \text{\qquad where $\epsilon^m_t \sim \mathsf{N}(0,\sigma^2), m \in [M]$.} $
\State Update the cluster parameters: ${\mb{\mu}}^*_i = \mathbf{B} {\mb{\mu}}^*_i + r_i \quad i \in [C_{t+1}]$
\EndIf
\EndFor
\end{algorithmic}
\end{algorithm}



\section{Supplementary Figures}

\begin{center}
\begin{figure}[h!]
\begin{subfigure}[b]{.3\textwidth}
\includegraphics[width=\textwidth]{../figs/new/pcaelements.pdf}
\caption{}
\label{fig:ICold}
\end{subfigure}
\begin{subfigure}[b]{.3\textwidth}
% \includegraphics[width=\textwidth]{../figs/new/ICclusteroldpca.pdf}
\caption{}
\label{fig:ICold}
\end{subfigure}
\begin{subfigure}[b]{.3\textwidth}
% \includegraphics[width=\textwidth]{../figs/new/ICclusteroldpca.pdf}
\caption{}
\label{fig:ICold}
\end{subfigure}
\caption{\jovo{dictionary: (a) from first 5 secs, (b) from all data, (c) spectrum from all data which is cumsum/sum} 
} \label{fig:timing}
\end{figure}
\end{center}

\begin{figure}[htbp]
	\centering
		% \includegraphics[height=3in]{../figs/new/pairs.pdf}
	\caption{(a) This shows the average number of true positives versus the average number of false positives in the intracellular cluster for 2 minute segments of the 4 minutes of the experiment.  \smug\ does better than    \jovo{let's make the symbols different for the different methods.  also, let's make the axes in terms of percentages, rather than raw numbers.} }
	\label{fig:asdf}
\end{figure}


\begin{figure}[htbp]
	\centering
		\includegraphics[height=3in]{../figs/new/pairs.pdf}
	\caption{caption}
	\label{fig:pairs}
\end{figure}



\begin{center}
\begin{figure}[h!]
\begin{subfigure}[b]{.2\textwidth}
\includegraphics[width=1\textwidth]{../figs/3dev}
\caption{}
\label{3dev}
\end{subfigure}
% \begin{subfigure}[b]{.28\textwidth}
% \includegraphics[width=\textwidth]{../figs/3devim/clus1}
% \caption{}
% \label{ex31}
% \end{subfigure}
% \begin{subfigure}[b]{.28\textwidth}
% \includegraphics[width=\textwidth]{../figs/3devim/clus2}
% \caption{}
% \label{ex32}
% \end{subfigure}
\begin{subfigure}[b]{.5\textwidth}
\includegraphics[width=\textwidth]{../figs/new/3chpca}
\caption{}
\label{3chpca}
\end{subfigure}
\caption{
Improving \smug\ by incorporating \emph{multiple} channels.
(a) Three electrode device showing local proximity of electrodes with channel indexes in large, red numbers. 
% (b,c) Top 2 most prevalent waveforms.  Each waveform shape is 2ms long.   Note in (a) we have a waveform that appears on both channel 2 and channel 3, whereas in (b) the waveform only appears in channel 3.  If only channel 3 was used, it would be difficult to separate the waveform in (a) and (b), as is demonstrated in Fig.\ 
(b) The representation of detected spikes on the 3rd channel in PCA space. This cluster does not seem separable here.
}
\end{figure}
\end{center}



\begin{figure}[htbp]
	\centering
		% \includegraphics[height=3in]{../figs/new/pairs.pdf}
	\caption{(a) intracellular waveform shape over time (b) and in PC space.}
	\label{fig:pairs}
\end{figure}



\begin{center}
\begin{figure}
	% \includegraphics[width=.5\textwidth]{../figs/IntracellularTrueFalsePositivesv2}
	% \includegraphics[width=.5\textwidth]{../figs/new/ICclusteroldpca.pdf}
		\begin{subfigure}[b]{.5\textwidth}
	\includegraphics[width=\textwidth]{../figs/IntracellularTrueFalsePositivesv2}
	\caption{}
	\label{truewaveforms}
	\end{subfigure}
\begin{subfigure}[b]{.5\textwidth}
\includegraphics[width=\textwidth]{../figs/new/ICclusteroldpca.pdf}
\caption{}
\label{fig:ICold}
\end{subfigure}
% \begin{subfigure}[b]{.33\textwidth}
% \includegraphics[width=\textwidth]{../figs/new/ICclusternewpca.pdf}
% \caption{}
% \label{fig:ICnew}
% \end{subfigure}
\caption{False and true positive detections have the same first-order statistics, making detection using only these statistics quite difficult.  (a)
 Errorbar plots of the true positives, false positives, and missed positives  in the IC cluster.  While the false positives have slightly more variability, the mean shape for the false positives and the true positives is nearly identical.  The true misses have a significantly lower amplitude as well as high variability. (b) All waveforms from the IC neuron as well as those we estimated from the IC neuron projected onto the first two PC space.} \label{fig:IC-PCA}
\end{figure}
\end{center}



\begin{center}
\begin{figure}
\begin{subfigure}[b]{.12\textwidth}
\includegraphics[width=0.8\textwidth]{../figs/8dev}
\caption{}
\label{8dev}
\end{subfigure}
\begin{subfigure}[b]{.28\textwidth}
\includegraphics[width=\textwidth]{../figs/8devim/clus3}
\caption{}
\label{ex81}
\end{subfigure}
\begin{subfigure}[b]{.28\textwidth}
\includegraphics[width=\textwidth]{../figs/8devim/clus9}
\caption{}
\label{ex82}
\end{subfigure}
\begin{subfigure}[b]{.28\textwidth}
\includegraphics[width=\textwidth]{../figs/8devim/clus6}
\caption{}
\label{ex83}
\end{subfigure}
\caption{
\smug\ multielectrode performance.
(a) 8 electrode device showing local proximity of electrodes with channel indexes in large, red numbers. (b,c,d) Top three most prevalent waveforms.  Each waveform shape is 2 ms long.
} \label{sfig:8}
\end{figure}
\end{center}



\begin{center}
\begin{figure}[h!]
\begin{subfigure}[b]{.5\textwidth}
\includegraphics[width=\textwidth]{../figs/new/timingsinglechannel.pdf}
\caption{}
\label{fig:ICold}
\end{subfigure}
\begin{subfigure}[b]{.5\textwidth}
% \includegraphics[width=\textwidth]{../figs/new/ICclusteroldpca.pdf}
\caption{}
\label{fig:ICold}
\end{subfigure}
\caption{\jovo{timing plots: (a) line (can we put a line on there for 4 channels too?), (b) notched boxplots. } 
} \label{fig:timing}
\end{figure}
\end{center}




\end{document}















